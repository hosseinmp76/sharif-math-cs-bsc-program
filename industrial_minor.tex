\documentclass[class=article, crop=false]{standalone}
\usepackage{hyperref}
\begin{document}
	\begin{table}[H]
	\begin{center}
		\begin{tabular}{|c|c|c|c|}
			\hline
			شماره & نام & پیشنیاز/همنیاز & تعداد واحد \\
			\hline
			21010 & آشنایی با مهندسی صنایع* &  & 0 \\
			\hline
			21131 & اقتصاد مهندسی & 22089 & 3 \\
			\hline
			21418 & روشهای تولید &  & 3 \\
			\hline
			21532 & برنامه ریزی و کنترل تولید & 22879 & 3 \\
			\hline
			21521 & کنترل کیفیت آماری & 22064 & 3 \\
			\hline
			21532 & کنترل پروژه & 22089 و 22879 & 3 \\
			\hline
			21612 & طرح ریزی واحدهای صنعتی &  & 3 \\
			\hline
		\end{tabular}
		\caption{\label{ind-t1}
		جدول شماره یک صنایع: دروس اصلی-الزامی
		}
	\end{center}
\end{table}

	* درس آشنایی با مهندسی صنایع در ابتدای نیمسالهای فرد و با آغاز دوره کهاد و قبل از ورود به دروس اصلی دانشجویان به صورت یک اردوی یک یا دو روزه جهت آشنایی با رشته مهندسی صنایع و با حضور اساتید و تحت نظارت دانشکده مهندسی صنایع و همراه با بازدید از سازمانهای تولیدی و خدماتی برگزار می شود.
\begin{table}[H]
\begin{center}
		\begin{tabular}{|c|c|c|c|}
			\hline
			شماره & نام &  & تعداد واحد \\
			\hline
			\multicolumn{4}{|c|}{الف- سبد برنامه ریزی و مدیریت} \\
			\hline
			21761 & برنامه ریزی حمل و نقل &  & 3 \\
			\hline
			21524 & برنامه ریزی تعمیر و نگهداری & 21131 & 3 \\
			\hline
			21644 & مبانی مدیریت زنجیره تامین & 21131 & 3 \\
			\hline
			\multicolumn{4}{|c|}{ب- سبد هوش تجاری} \\
			\hline
			21019 & مبانی داده کاوی و کاربردها &  & 3 \\
			\hline
			21942 & اصول شبیه سازی &  & 3 \\
			\hline
			21780 & مبانی هوش تجاری &  & 3 \\
			\hline
			\multicolumn{4}{|c|}{ج- سبد تولید و مدیریت محصول} \\
			\hline
			21416 & مقدمات سیالات محاسباتی & 21418 & 3 \\
			\hline
			21423 & برنامه ریزی و توسعه محصول & 21418 & 3 \\
			\hline
			21430 & تولید ناب & 21418 & 3 \\
			\hline
		\end{tabular}
		\caption{\label{ind-t2}
		جدول شماره دو صنایع: دروس سبدهای انتخابی
		}
	\end{center}
\end{table}

\end{document}