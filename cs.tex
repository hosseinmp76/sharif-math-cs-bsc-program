\documentclass{article}
\usepackage[utf8]{inputenc}
\usepackage[subpreambles=true]{standalone}
\usepackage{import}
\usepackage{amsthm}
\usepackage{pgffor}
\usepackage{latexsym}
\usepackage{xargs}
\usepackage{xparse}
\usepackage{tabularx}
\usepackage{etoolbox}
\usepackage{amssymb}
\usepackage{verbatim}
\usepackage{float}
\usepackage{enumitem,amsmath,array}
\usepackage{tikz}
\usepackage{tkz-graph}
\usepackage{bookmark}
\usetikzlibrary{positioning,chains,fit,shapes,calc}
\usepackage[a4paper, margin=1.0in]{geometry}
\usepackage{listings}
\usepackage{pgf}
\usepackage{pgfpages}
\usepackage{hyperref}
\usepackage{xifthen}% provides \isempty test
\usepackage{ragged2e}
\usepackage{graphicx}
\usepackage{blindtext}
\usepackage{mathrsfs}


% add your packeges here.
% this is added for s002
\usepackage{mathtools}
% this is added for s002
\usepackage{subcaption}
% this is added for s002

\graphicspath{{images/}}
\hypersetup{
    colorlinks=true,
    linkcolor=blue,
    filecolor=magenta,
    urlcolor=cyan,
}

% below packages have come from here :https://tex.stackexchange.com/questions/287131/which-encoding-used-in-latex-sources-to-convert-pdf-file-accents-to-the-doc-form
\usepackage{lmodern}
\usepackage[T1]{fontenc}



\usepackage{background}
\usetikzlibrary{calc}
\backgroundsetup{angle = 0, scale = 1, vshift = -2ex,
	contents = {\tikz[overlay, remember picture]
		\draw [rounded corners = 20pt, line width = 1pt,
		color = blue, double = black!10]
		($(current page.north west)+(-7cm,11cm)$)
		rectangle ($(current page.south east)+(-9cm,15cm)$);
	}
	}
\pagestyle{empty}
\usepackage{import}
\usepackage{amsthm}
\usepackage{latexsym}
\usepackage{tabularx}
\usepackage{amssymb}
\usepackage{verbatim}
\usepackage{float}
\usepackage{enumitem,amsmath,array}
\usepackage{tikz}
\usepackage{tkz-graph}
\usepackage{bookmark}
\usetikzlibrary{positioning,chains,fit,shapes,calc}
\usepackage[a4paper, margin=1.0in]{geometry}
\usepackage{listings}
\usepackage{pgf}
\usepackage{pgfpages}
\usepackage{hyperref}
\usepackage{xifthen}% provides \isempty test
\usepackage{ragged2e}
\usepackage{graphicx}
\usepackage{blindtext}
\usepackage{mathrsfs}


\newcommand{\loopy}[6]{
    \fbox{\parbox{\textwidth}{
    		\hypertarget{#1}{
        نام درس: #1
    }
        \\
        شمارهی درس: #2
        \\
        تعداد واحد: #3
        \\پیشنیاز: #4
    }}
    \\
    \ifthenelse{\isempty{#5}}
    {}{
        \fbox{\parbox{\textwidth}{
            اهداف آموزشی:
            \begin{enumerate}
                \foreach \x in #5 {
                    \item \x
                }
            \end{enumerate}
        }}
    }
    \ifthenelse{\isempty{#6}}
    {}{
        \fbox{\parbox{\textwidth}{
            ریز مواد:
            \begin{enumerate}
                \foreach \x in #6 {
                    \item \x
                }
            \end{enumerate}
        }
        }
    }
}

\begingroup
\catcode`@=11
\long\gdef\firstofmany#1{%
	\@fom{\unexpanded{[#1]}}#1{[#1]}}

\long\gdef\@fom#1#2{%
	\unexpanded{#2}%
	\@gobbleto{#1}}

\gdef\@gobbleto#1#2{%
	\ifnum\pdfstrcmp{\unexpanded{#2}}{#1}=\z@
	\expandafter\@gobbletwo
	\else
	\fi
	\@gobbleto{#1}}

\gdef\@gobbletwo#1#2{}

\endgroup


% Example usage
\newcommand{\Cone}[6]{%
	% Code for C1 command using the 6 arguments #1, #2, #3, #4, #5, #6
	% ...
	% ...
	% ...
	\pagebreak
	\loopy{#1}{#2}{#3}{#4}{#5}{#6}
	
}

\newcommand{\callConeList}[1]{
  	\begin{enumerate}
		\renewcommand*{\do}[1]{%
			\item \hyperlink{\firstofmany{##1}}{\firstofmany{##1}}
		}
		\docsvlist{#1}% Process the list to create an enumerate environment
	\end{enumerate}
    \renewcommand*{\do}[1]{\Cone##1}% Call C1 with 6 arguments from each element of the list
    \docsvlist{#1}% Process the list
}




%\usepackage[fontsloadable]{xepersian}
\usepackage{xepersian}

\settextfont[
    Scale=1.09,
    Extension=.ttf,
    Path=./font/,
    ItalicFont=*It,
    BoldFont=*Bd,
    BoldItalicFont=*BdIt,
]{HM XNiloofar}

\title{
    برنامه‌ی مقطع کارشناسی رشته‌ی علوم کامپیوتر
}
\author{
    دانشکده علوم ریاضی
}
\date{
    تیر 1402
}


\begin{document}
    \maketitle
    \section{مقدمه }
    این برنامه بر اساس آخرین برنامهی مصوب رشته های علوم ریاضی در وزارت علوم، تحقیقات و فناوری، و با توجه به شرایط فعلی دانشکدهی علوم ریاضی و رشته ی علوم کامپیوتر، تدوین شده است. بدیهی است برنامه می تواند در بخشهایی که در اختیار دانشکده است، بر حسب نیازهای جاری و آتی تغییر کند و مجدداً جهت اجرا به دانشجویان ابلاغ شود. این برنامه برای دانشجویان ورودی 1401 و ما بعد لازالاجرا و برای دانشجویان 1400 و ما قبل پس از تأیید استاد راهنما و دانشکده، قابل اجرا است.
    
\pagebreak
\tableofcontents
\pagebreak
\listoftables
\newpage
\section{ساختار کلّی برنامه}
    برنامه شامل بخشهای زیر است:
\begin{table}[H]
\begin{center}
        \begin{tabular}{|c | c | c|}
            \hline
            {دروس عمومی} & & {20 واحد} \\
            \hline
            {دروس الزامی دانشگاه (پایه)}\LR{)} & {
            \href{cs-t1}{
            	جدول 
            }
            } & {23 واحد} \\
            \hline
            {دروس الزامی دانشکده} & {
            \href{cs-t2}{
            	جدول 
            }
            } & {16 واحد} \\
            \hline
            {دروس الزامی-تخصصی} & {
            	\href{cs-t3}{
            		جدول 
            	}
            	} & {33 واحد} \\
            \hline
            {دروس انتخابی-تخصصی} & & {30 واحد} \\
            \hline
            {دروس اختیاری(خارج دانشکده) } & & {12 واحد} \\
            \hline
            {جمع کل واحدها} & & {134 واحد} \\
            \hline
        \end{tabular}
		\caption{\label{cs-t1}
	ساختار کلی برنامه
		}
    \end{center}
\end{table}
    \section{مقررات برنامه}
    \begin{enumerate}
        \item
        مقررات اخذ 20 واحد دروس عمومی تابع ضوابط تعیین شده توسط معاونت آموزشی و تحصیلات تکمیلی در دانشگاه است و دانشجو ملزم به اخذ و گذراندن دروس مربوطه در چارچوب ضوابط تعیین شده خواهد بود.
        \item
        اخذ و گذراندن تمامی دروس جدولهای
        \ref{cs-t1}
        ،
        \ref{cs-t2}
        و همچنین
        \ref{cs-t3}
         توسط دانشجو الزامی است.
        \item
        اخذ و گذراندن 30 واحد از دروس «انتخابی-تخصصی» با رعایت مقررات زیر الزامی است:
        \begin{enumerate}
            \item
            انتخاب حداقل چهار «زمینه تخصصی» متفاوت و گذراندن حداقل 3 واحد در هر یک از این چهار زمینه برای دانش آموختگی در این برنامه الزامی است. زمینه های تخصصی این برنامه در تاریخ تصویب شامل موارد زیر است:
            \begin{enumerate}
                \item
                نظریه محاسبه و الگوریتم
                \item
                 محاسبات علمی
                \item
                 نظریه سیستمها
                \item
                 علوم داده
                \item
                 محاسبات نرم و هوش مصنوعی
                \item
                 محاسبات زیستی
                \item
                 کدگذاری و رمزنگاری
            \end{enumerate}
            \item 
            زمینه تخصصی دروس انتخابی-تخصصی، از بین زمینه های تخصصی بالا، توسط دانشکده علوم ریاضی تعیین و اعلام می شود.
            \item
             سایر واحدهای باقیمانده در بخش «دروس انتخابی-تخصصی» می تواند با گذراندن هر یک از دروس ارائه شده در دانشکده علوم ریاضی (از جمله هر یک از دروس مقطع کارشناسی رشته های علوم کامپیوتر و ریاضیات و کاربردها، دروس مقطع تحصیلات تکمیلی، سمینار یا پروژه کارشناسی) با رعایت مقررات آموزشی و تایید استاد راهنما تکمیل شود.
        \end{enumerate}
        \item 
        در بخش  «دروس اختیاری (خارج دانشکده)»، اخذ دروس با مشورت و تایید استاد راهنما صورت می پذیرد و این دروس الزاما باید از دروس خارج از دانشکده علوم ریاضی انتخاب شوند. 
        \begin{enumerate}
        	\item
        
    تبصره 1: درچارچوب قوانین و مقررات آموزشی دانشگاه، گذراندن حداکثر شش واحد از دروس مراکز (معارف، زبانها، کارگاه ها) و گروه ها (فلسفه علم) مجاز است. سایر واحدها باید از دروس تخصصی دانشکده های دیگر اخذ شود.
    \item
    تبصره 2:  گذراندن حداقل 3 واحد با موضوع مدیریت و اقتصاد با تایید استاد راهنما الزامی است.
	\end{enumerate}
        \item
         سرفصل تمامی دروس، مطابق سرفصل درج شده در این برنامه یا بر اساس آخرین سرفصل اعلامی توسط دانشکده علوم ریاضی در مورد هر درس تعیین می شود.
        \item
         دانش آموختگی در این برنامه با شرکت در یک برنامه کهاد در چارچوب مقررات مندرج در بخش (4) این برنامه نیز برای هر دانشجو امکانپذیر است ( الزامی نیست). برای شرکت هر دانشجو در یک برنامه کهاد، تایید دانشکده علوم ریاضی و دانشکده مقصد (یعنی دانشکده ارائه دهنده دوره کهاد مصوب مربوطه برای رشته علوم کامپیوتر) الزامی است. مجموع واحدهای دانشجویی که برنامه دوره خود را با یک کهاد مصوب به پایان میرساند می تواند تا سقف 137 واحد افزایش داشته باشد.
        \item مقررات اخذ دروس مازاد بر مجموع واحدهای دوره، مطابق قوانین ابلاغی از طرف اداره کل آموزش دانشگاه مشخص می شود.
    \end{enumerate}

	\section{
	جداول دروس رشته
}
شماره های دروس، مطابق شمارهی دروس مربوطه در سیستم آموزش دانشگاه است.
\begin{table}[H]
\begin{center}
	\begin{tabular}{|c|c|c|c|}
		\hline
		شماره & نام & پیشنیاز/همنیاز & تعداد واحد  \\
		\hline
		22015 & ریاضیات عمومی 1 &  & 4 \\
		\hline
		22016 & ریاضیات عمومی 2 & 22015 & 4 \\
		\hline
		22034 & معادلات دیفرانسیل & همنیاز با 22016 & 3 \\
		\hline
		22048,22049 & مبانی کامپیوتر و برنامهنویسی &  & *3+1 \\
		\hline
		24011 & فیزیک 1 &  & 3 \\
		\hline
		24001 & آز فیزیک 1 & همنیاز با 24011 &  \\
		\hline
		24012 & فیزیک 2 & 24011 &  \\
		\hline
		33018 & کارگاه عمومی &  & 1 \\
		\hline
	\end{tabular}
\caption{\label{cs-t2}
دروس الزامی دانشگاه (پایه)
}
\end{center}
\end{table}
* این درس بصورت سه واحد نظری و یک واحد عملی ارائه می شود.
\begin{table}[H]
\begin{center}
\begin{tabular}{|c|c|c|c|}
	\hline
	شماره & نام & پیشنیاز/همنیاز & تعداد واحد \\
	\hline
	22255 & جبر خطی 1 & 22016 & 4 \\
	\hline
	22089 & احتمال و کاربرد آن & 22016 & 4 \\
	\hline
	22655 & آنالیز عددی 1 & 22016 & 4 \\
	\hline
	22142 & مبانی ریاضیات &  & 4 \\
	\hline
\end{tabular}
\caption{\label{cs-t3}
دروس الزامی دانشکده
}
\end{center}
\end{table}

\begin{table}[H]
\begin{center}
\begin{tabular}{|c|c|c|c|}
	\hline
	شماره & نام & پیشنیاز/همنیاز & تعداد واحد \\
	\hline
	22825 & ریاضیات گسسته &  22015 & 3 \\
	\hline
	22067 & آمار و کاربرد آن & 22089 & 3 \\
	\hline
	22131 & منطق ریاضی* & 22142 & 3 \\
	\hline
	22834 & برنامهنویسی پیشرفته & 22049 & 3 \\
	\hline
	22900 & اصول سیستمهای کامپیوتری & 22815 & 3 \\
	\hline
	22688 & ساختمان داده ها & 22815 & 3 \\
	\hline
	22873 & نظریهی زبانها و اتوماتا & 22825 & 3 \\
	\hline
	22861 & سیستم عامل 1 & 22885 هم نیاز & 3 \\
	\hline
	22499 & بهینهسازی خطی* & 22255 & 3 \\
	\hline
	22891 & آنالیز الگوریتمها & 22822 & 3 \\
	\hline
	22035 & ریاضی مهندسی** & 22034 & 3 \\
	\hline
\end{tabular}
\caption{\label{cs-t4}
دروس الزامی تخصصی
}
\end{center}
\end{table}
* درصورت اخذ این درس بصورت چهارواحدی، واحد مازاد در جدول دروس تخصصی انتخابی قابل تطبیق است.
** دانشجویان می توانند به جای درس ریاضی مهندسی یکی از دروس توابع مختلط 1 (22335) یا معادلات دیفرانسیل با مشتقات جزئی (22395) را بگذرانند که در ایندرصورت، یک واحد مازاد در جدول دروس تخصصی انتخابی قابل تطبیق است. گذراندن هر سه درس مذکور مجاز نیست. 

\section{برنامه کهاد}
به منظور افزایش و تشویق آموزشهای بین رشته ای و در راستای برنامه ها و توصیه های وزارت عتف در برگزاری دوره های کهاد و به استناد برنامه دوره کارشناسی علوم ریاضی مصوب 26/2/1388، امکان انتخاب یک برنامه کهاد (شامل حداقل 24 و حداکثر 27 واحد درسی) توسط دانشجویان از بین برنامه های کهاد مصوب در دانشگاه برای رشته علوم کامپیوتر وجود دارد. در اینصورت دانشجوی علوم کامپیوتر با گذراندن یک بسته آموزشی کهاد می تواند با عنوان مدرک کارشناسی در رشته علوم کامپیوتر و کهاد مربوطه دانش آموخته شود. بدین منظور پس از تصویب و دریافت مجوز ادامه تحصیل در یک کهاد مصوب برای یک دانشجو از دانشکده های مبدا و مقصد، مجموعه دروس کهاد مصوب مربوطه در قالب «دروس اختیاری (خارج از دانشکده)» و نیز بخشی از «دروس انتخابی-تخصصی» قابل تطبیق است. سقف تعداد واحدهای قابل تطبیق در قالب این جداول 24 واحد است که با اولویت ابتدا در دروس اختیاری خارج از دانشکده و سپس جدول دروس انتخابی-تخصصی انجام خواهد شد و واحدهای مازاد بر 24 واحد دوره کهاد باعث افزایش تعداد واحدهای لازم برای فارغ التحصیلی تا سقف 137 واحد خواهد شد و واحدهای افزایش یافته مازاد محسوب نخواهد شد. در تطبیق دروس دوره کهاد، رعایت تبصره 2 بند 4 مربوط به جدول دروس اختیاری (خارج از دانشکده) و بند 3-الف مربوط به جدول دروس انتخابی-تخصصی در بخش 3 الزامی است. 

شرایط و مقررات عمومی کهاد
\begin{enumerate}
    \item
     دوره کهاد یک دوره اختیاری در برنامه علوم کامپیوتر است که به جهت ایجاد فضای میانرشته ای و آموزش دانش آموختگانی با علائق و تواناییهای متنوع و ترکیبی طراحی شده است. لذا، علاقه دانشجو به رشته کهاد مورد نظر و دقت در انتخاب آن و همچنین بهره گیری از مشاوره لازم در طی دوره توسط دانشجو از ضروریات موفقیت در این دوره است. دوره کهاد یک امکان و انتخاب در برنامه با اهداف خاص است و پذیرش دانشجو در این دوره ها اساسا به صورت بسیار محدود و با دقت نظر لازم توسط دانشکده علوم ریاضی و دانشکده مقصد (دانشکده ارائه دهنده دوره کهاد) صورت می پذیرد.
	\item
	 درخواست دانشجو برای شرکت در برنامه کهاد باید پس از پایان نیمسال چهارم ارائه شده و حداکثر تا قبل از شروع نیمسال ششم تحصیلی تعیین تکلیف و نتیجه درخواست نهایی شود.
	\item
	 برای تایید درخواست دانشجو مبنی بر شرکت و ادامه تحصیل در دوره کهاد در برنامه علوم کامپیوتر، موافقت دانشکده علوم ریاضی، موافقت دانشکده مقصد (یعنی دانشکده ارائه دهنده دوره کهاد) و رعایت کلیه مقررات و آیین نامه های دانشکده مقصد و اداره کل آموزش دانشگاه در ارتباط با دوره کهاد مربوطه الزامی است.
	\item
	 استاد راهنمای دوره کهاد معاون آموزشی دانشکده مقصد یا یک نفر از اساتیدی است که توسط دانشکده مقصد تعیین می شود و دانشجو برنامه کهاد خود را زیر نظر ایشان و با رعایت کلیه مقررات دانشکده مقصد و دانشکده علوم ریاضی دنبال خواهد کرد. مشاوره با معاون آموزشی یا اساتید متخصص در دانشکده مقصد قبل از ارائه درخواست توسط دانشجو و همچنین در طی دوره موکدا توصیه می شود.
	\item
	 برنامه های کهاد مصوب دانشگاه در رشته علوم کامپیوتر اساسا با تایید و نظارت معاونت آموزشی دانشگاه اجرا می شود و اخذ کلیه مجوزهای لازم در هر مرحله و رعایت کلیه مقررات جاری اداره کل آموزش دانشگاه در ارتباط با ادامه تحصیل در دوره کهاد مربوطه الزامی است.
	\item
	 با توجه به مقررات و در صورت وجود مجوزهای لازم، امکان درج نام کهاد مربوطه در دانشنامه پایان تحصیلات مقطع کارشناسی وجود خواهد داشت و پس از پایان موفقتیت آمیز دوره کارشناسی علوم کامپیوتر، دانشجویی که رشته علوم کامپیوتر را با یک کهاد مصوب به پایان برساند می تواند در رشته علوم کامپیوتر و با درج نام کهاد مربوطه در دانشنامه خود دانش آموخته شود.
	\item
	 رعایت کلیه مقررات آموزشی دانشگاه و وزارت علوم، تحقیقات و فناوری، از جمله سقف سنوات تحصیلی، سقف و کف واحدهای مجاز در هر نیمسال، رعایت پیشنیازی و همنیازی دروس و ضوابط مشروطی در طول دوره تحصیل دانشجو الزامی است و دانشجوی پذیرفته شده در یک دوره کهاد همانند سایر دانشجویان ملزم به رعایت تمامی مقررات مرتبط با دوره تحصیلی خود خواهد بود.
\end{enumerate}

\section{برنامه کهادهای مصوب}
کهادهای زیر در زمان تصویب این برنامه با هماهنگی و تایید دانشکده های ذیربط و تایید دانشگاه تعیین و تصویب شده اند و از تاریخ تصویب این برنامه با رعایت تمامی مقررات مربوطه قابل اجرا هستند. دانشکده علوم ریاضی می تواند با هماهنگی با دانشکده های دیگر و تایید دانشگاه، نسبت به بازنگری برنامه های کهاد موجود یا اضافه کردن برنامه های کهاد جدید اقدام و پس از تصویب در دانشگاه به این مجموعه جهت اجرا اضافه کند. 
این دوره ها به عنوان يك دوره آموزشي اختياري براي دانشجويان رشته علوم کامپیوتر دانشكده علوم ریاضی كه علاقه مند به آشنایی و فراگیری ابعادی از رشته کهاد مورد نظر که مرتبط با و یا مکمل رشته تحصیلی خود باشند طراحی شده است. دوره های کهاد خصوصاً براي دانشجویانی كه قصد دارند در زمینه هاي بین رشته ای و نوين علم و فناوري ادامه فعالیت دهند، بسيار مناسب است.


\subsection{
کهاد مهندسی مکانیک
}
دوره کهاد مهندسي مكانيك شامل 24 واحد درسی است که 18 واحد آن الزاما باید از بین مجموعه دروس اصلی-الزامی این کهاد (
\href{mech-t1}{جدول اول مکانیک}
) و 6 واحد باقیمانده منطبق با اهداف نهايي دانشجو، صرفا از بین یکی از سه سبد انتخابی این کهاد (
\href{mech-t2}{جدول دوم مکانیک}
) انتخاب شود.
\subsection{
کهاد اقتصاد
}
دوره کهاد اقتصاد شامل 24 واحد درسی است که  12 واحد آن باید مطابق جدول 
\href{eco-t1}{جدول اول اقتصاد}
 (دروس اصلی-الزامی کهاد) و 6 واحد آن از بین مجموعه دروس انتخابی-الزامی این دوره (
 \href{eco-t2}{جدول دوم اقتصاد}
 ) اخذ شوند. شش واحد باقیمانده می تواند از مابقی دروس انتخابی-الزامی ( 
 \href{eco-t2}{جدول دوم اقتصاد}
 )
 (که قبلا به عنوان درس انتخابی تطبیق نشده باشند) یا هر یک از دروس 
 \href{eco-t3}{جدول سوم اقتصاد}
 انتخاب شوند. برای دانشجویانی که در این کهاد پذیرفته می شوند، یکی از دروس دوره کهاد به عنوان درس مورد نیاز برای ارضای تبصره 2 بند 4 در بخش 3 (گذراندن حداقل 3 واحد در زمینه مدیریت یا اقتصاد) تطبیق خواهد شد.
\subsection{
کهاد مهندسی صنایع
}
دوره کهاد مهندسي صنایع شامل 24 واحد درسی است که 18 واحد آن الزاما باید از بین مجموعه دروس اصلی-الزامی این کهاد (
\href{ind-t1}{جدول اول صنایع}
) و 6 واحد باقیمانده منطبق با اهداف نهايي دانشجو، صرفا از بین یکی از سه سبد انتخابی این کهاد (
\href{ind-t2}{جدول دوم صنایع}
)
 انتخاب شود.
\subsection{
کهاد ریاضیات و کاربردها
}
دوره کهاد ریاضیات و کاربردها شامل 24 واحد درسی است که 8 واحد آن شامل دروس جبر 1 (22217) و آنالیز ریاضی 1 (22325) است و 16 واحد باقیمانده، باید از بین سایر دروس الزامی-تخصصی یا انتخابی-تخصصی گرایش ریاضی محض (که دانشجو در رشته خود نگذرانیده است) انتخاب شود.

\section{جداول دروس برنامه های کهاد}

\import{./}{mechanics_minor}
\import{./}{economics_minor}
\import{./}{industrial_minor}

\pagebreak
\section{
برنامهی ترمی دروس (پیشنهادی)
}
این برنامهی ترمی، کاملاً پیشنهادی و به عنوان نمونه فقط برای یک دانشجوی نوعی ارائه شده است. پیشنهاد می شود که دانشجویان حجم دروس هر نیمسال و جزئیات چگونگی اخذ دروس خود را با توجه به شرایط خود و دروس ارائه شده در هر نیمسال تحصیلی، با استاد راهنمای دوره هماهنگ و سپس نسبت به اخذ دروس اقدام کنند.
برای پیگیری و ضبط فرایند تحصیلی شما پیشنهاد می شود:
\begin{enumerate}
	

\item
 زمینهی دروس انتخابی-تخصصی خود را در ستون ملاحظات درج کنید.
\item
 در صورت اخذ دروس در نیمسالهای دیگر، روی این دروس در برنامه، خط بکشید و در نیمسال مربوط، نام، مشخصات و نمرهی دروس گذراندهشده را وارد کنید.
\item
 فایل اصلی این برنامه، از طریق وب-سایت دانشکده قابل دسترسی است.
\end{enumerate}

\begin{table}[H]
 \begin{center}
 \begin{tabular}{|c|p{9cm}|c|}
 	\hline
 	نیمسال & 
 			دروس پیشنهادی 

 	& تعداد واحد \\
 	\hline
 	اول &  & 16 \\
 	\hline
 	دوم &  & 17 \\
 	\hline
 	سوم &  & 17 \\
 	\hline
 	 چهارم &  & 18 \\
 	\hline
 	پنجم &  & 17 \\
 	\hline
 	ششم &  & 17 \\
 	\hline
 	هفتم &  & 16 \\
 	\hline
 	هشتم &  & 16 \\
 	\hline
 \end{tabular}
\caption{\label{cs-t5}
نمای کلی برنامه درسی در هر ترم
}
\end{center}
\end{table}

\begin{table}[H]
\begin{center}
\begin{tabular}{|c|c|c|c|}
	\hline
	ملاحظات & واحد & عنوان درس & شناسه‌ی درس \\
	\hline
	& 4 & ریاضی عمومی 1 & 22015 \\
	\hline
	& 3 & فیزیک 1 & 24011 \\
	\hline
	& 1 & آز فیزیک 1 & 24001 \\
	\hline
	& 3+1 & مبانی کامپیوتر و برنامهنویسی &  \\
	\hline
	& 1 & کارگاه عمومی & 33018 \\
	\hline
	& 3 & از سبد عمومی &  \\
	\hline
	&  &  &  \\
	\hline
		&  &  &  \\
	\hline
	\multicolumn{4}{|c|}{} \\
	\hline
\end{tabular}
\caption{\label{cs-t6}
برنامه پیشنهادی نیم‌سال اول
}
\end{center}
\end{table}

\begin{table}[H]
\begin{center}
	\begin{tabular}{|c|c|c|c|}
		\hline
		شناسه‌ی درس & عنوان درس & واحد & ملاحظات \\
		\hline
		22016 & ریاضی عمومی 2 & 4 &  \\
		\hline
		24012 & فیزیک 2 & 3 &  \\
		\hline
		22815 & برنامه نویسی پیشرفته & 3 &  \\
		\hline
		& مبانی ریاضیات & 4 &  \\
		\hline
		& از سبد عمومی & 3 &  \\
		\hline
		& جمع واحدها & 17 &  \\
		\hline
	\end{tabular}
\caption{\label{cs-t7}
برنامه پیشنهادی نیم‌سال دوم
}
	\end{center}
	\end{table}

\begin{table}[H]
\begin{center}
\begin{tabular}{|c|c|c|p{0.5\linewidth}|}
	\hline
	شناسه‌ی درس & عنوان درس & واحد & ملاحظات \\
	\hline
	22885 & اصول سیستم های کامپیوتری & 3 &  \\
	\hline
	22089 & احتمال و کاربرد آن & 4 &  \\
	\hline
	22255 & جبر خطی 1 & 4 &  \\
	\hline
	22822 & ساختمان داده & 3 &  \\
	\hline
	22825 & ریاضیات گسسته & 3 & 
	یا یک از دروس نظریه گراف و کاربرد آن (22162) یاترکیبیات و کاربردهای آن (2218)
 \\
	\hline
	& جمع واحدها & 17 &  \\
	\hline
\end{tabular}
\caption{\label{cs-t8}
برنامه پیشنهادی نیم‌سال سوم
}
\end{center}
\end{table}

\begin{table}[H]
\begin{center}
	\begin{tabular}{|c|c|c|c|}
		\hline
		شناسه‌ی درس & عنوان درس & واحد & ملاحظات \\
		\hline
		22861 & سیستم عامل 1 & 3 &  \\
		\hline
		22891 & آنالیز الگوریتمها & 3 &  \\
		\hline
		22655 & آنالیز عددی 1 & 4 &  \\
		\hline
		22034 & معادلات دیفرانسیل & 3 &  \\
		\hline
		22064 & آمار و کاربرد آن & 3 &  \\
		\hline
		& از سبد عمومی & 2 &  \\
		\hline
		& جمع واحدها & 18 &  \\
		\hline
	\end{tabular}
\caption{\label{cs-t9}
برنامه پیشنهادی نیم‌سال چهارم
}
\end{center}
		\end{table}

		\begin{table}[H]
\begin{center}
\begin{tabular}{|c|c|c|p{0.5\linewidth}|}
	\hline
	شناسه‌ی درس & نظریهی زبانها و اتوماتا & 3 &  \\
	\hline
	22873 & منطق ریاضی & 3 &  \\
	\hline
	22131 & ریاضی مهندسی & 3 & یا یکی از دروس توابع مختلط (22041) یا آشنایی با معادلات دیفرانسیل با مشتقات جزیی (22042) \\
	\hline
	22035 & بهینهسازی خطی & 3 &  \\
	\hline
	& از سبد انتخابی/اختیاری & 3 &  \\
	\hline
	& از سبد عمومی & 2 &  \\
	\hline
	& جمع واحدها & 17 &  \\
	\hline
\end{tabular}
\caption{\label{cs-t10}
برنامه پیشنهادی نیم‌سال پنجم
}
\end{center}
\end{table}

\begin{table}[H]
\begin{center}
\begin{tabular}{|c|c|c|c|}
	\hline
	شناسه‌ی درس & عنوان درس & واحد & ملاحظات \\
	\hline
	& از سبد انتخابی/اختیاری & 10 &  \\
	\hline
	& از سبد عمومی & 4 &  \\
	\hline
	& از زمینه مدیریت و اقتصاد & 3 &  \\
	\hline
	& 17 & جمع واحدها & \\
	\hline
\end{tabular}
	\caption{\label{cs-t11}
برنامه پیشنهادی نیم‌سال ششم
}
\end{center}
\end{table}
\begin{table}[H]
\begin{center}
\begin{tabular}{|c|c|c|c|}
	\hline
	شناسه‌ی درس & عنوان درس & واحد & ملاحظات \\
	\hline
	& از سبد انتخابی/اختیاری & 14 &  \\
	\hline
	& از سبد عمومی & 2 &  \\
	\hline
	&  &  &  \\
	\hline
	&  &  &  \\
	\hline
	&  &  &  \\
	\hline
	&  &  &  \\
	\hline
	&  &  &  \\
	\hline
	&  &  &  \\
	\hline
	& جمع واحدها & 16 &  \\
	\hline
\end{tabular}
\caption{\label{cs-t12}
برنامه پیشنهادی نیم‌سال هفت
}
\end{center}
\end{table}
\begin{table}[H]
\begin{center}
\begin{tabular}{|c|c|c|c|}
	\hline
	شناسه‌ی درس & عنوان درس &     واحد    & ملاحظات \\
	\hline
	& از سبد انتخابی/اختیاری & 12 &  \\
	\hline
	& از سبد عمومی & 4 &  \\
	\hline
	& جمع واحدها & 16 &  \\
	\hline
\end{tabular}
\caption{\label{cs-t13}
برنامه پیشنهادی نیم‌سال هشتم
}
\end{center}
\end{table}

\pagebreak
\import{./}{courses}
\end{document}