\documentclass[class=article, crop=false]{standalone}
\usepackage{hyperref}
\begin{document}

	\begin{table}[H]
		\begin{center}
		\begin{tabular}{|c|c|c|c|}
			\hline
			شماره & نام & پیشنیاز/همنیاز & تعداد واحد \\
			\hline
			44714 & مبانی اقتصاد & - & 3 \\
			\hline
			44719 & اقتصاد خرد & 44714 & 3 \\
			\hline
			44728 & اقتصاد کلان & 44714 & 3 \\
			\hline
			44720 & اقتصادسنجی & 44714 همنیاز & 3 \\
			\hline
		\end{tabular}
		\caption{\label{eco-t1}
		جدول شماره یک اقتصاد: دروس اصلی-الزامی
		}
	\end{center}
	\end{table}
	\begin{table}[H]
	\begin{center}
		\begin{tabular}{|c|c|c|c|}
			\hline
			شماره & نام & پیشنیاز/همنیاز & تعداد واحد \\
			\hline
			44737 & اقتصاد مالی & 44714 & 3 \\
			\hline
			44625 & اصول نظریه بازی & 44719 & 3 \\
			\hline
			44748 & یادگیری ماشین در مالی & 44737 & 3 \\
			\hline
			44767 & معادلات الگوریتمی & 44737 & 3 \\
			\hline
		\end{tabular}
		\caption{\label{eco-t2}
		جدول شماره دو اقتصاد: دروس انتخابی-الزامی
		}
	\end{center}
	\end{table}
	\begin{table}[H]
	\begin{center}
		\begin{tabular}{|c|c|c|c|}
			\hline
			شماره &  نام  &  پیشنیاز/همنیاز  & 	تعداد واحد  \\
			\hline
			44733 &  رشد اقتصادی  &  44728  & 	3  \\
			\hline
			44738 &  آشنایی با اقتصاد توسعه  &   44719 و 44720  & 	3  \\
			\hline
			44624 &  آشنایی با اقتصاد سیاسی  &  44728  & 	3  \\
			\hline
			44698 &  آشنایی با تجارت بین الملل  &  44719  & 	3  \\
			\hline
			&  آشنایی با اقتصاد بخش عمومی  &  44728  & 	3  \\
			\hline
			44741 &  آشنایی با سازماندهی صنعتی  &  44714  & 	3  \\
			\hline
			44739 &  آشنایی با پول و بانک  &  44714  & 	3  \\
			\hline
			44713 &  اقتصاد ایران  &  44728  & 	3  \\
			\hline
		\end{tabular}
		\caption{\label{eco-t3}
		جدول شماره سه اقتصاد: دروس انتخابی-اختیاری
		}
	\end{center}
	\end{table}
\end{document}