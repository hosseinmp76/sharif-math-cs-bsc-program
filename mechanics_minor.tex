\documentclass[class=article, crop=false]{standalone}
\usepackage{hyperref}
\begin{document}
	\begin{table}[H]
\begin{center}
\begin{tabular}{|c|c|c|c|}
	\hline
	شماره & نام & پیشنیاز/همنیاز & تعداد واحد \\
	\hline
	28261 & استاتيك * \# & - & 3 \\
	\hline
	28567 & ديناميك * & 28261  & 4 \\
	\hline
	28262 & مقاومت مصالح 1 * & 28261 & 3 \\
	\hline
	28263 & مقاومت مصالح 2 & 28262 & 2 \\
	\hline
	28861 & علم مواد     & - & 3 \\
	\hline
	28568 & ارتعاشات * & 28567  & 3 \\
	\hline
	28416 & کنترل اتوماتیک * & 28568 & 3 \\
	\hline
	28846 & الکترونیک عملی * & - & 3 \\
	\hline
	28233 & آز الکترونیک عملی * & 28846 هم نیاز & 1 \\
	\hline
	28461 & مكانيك سيالات1 \# & 28261 & 3 \\
	\hline
	28462 & مکانیک سیالات2  \# & 28461  & 3 \\
	\hline
	28161 & ترموديناميك1 \# & 28261 & 3 \\
	\hline
	28162 & ترمودینامیک 2 \# & 28161 & 3 \\
	\hline
	28113 & انتقال حرارت 1 \# & 28161 هم نیاز & 3 \\
	\hline
\end{tabular}
\caption{\label{mech-t1}
جدول شماره یک مکانیک: دروس اصلی-الزامی
}
\end{center}
	\end{table}
* برای دانشجویانی که سبد رباتیک را انتخاب میکنند توصیه می شود.

\text{\#}
برای دانشجویانی که سبدهای تبرید-تهویه و مکانیک سیالات محاسباتی را انتخاب میکنند توصیه می شود.
\begin{table}[H]
	\begin{center}
\begin{tabular}{|c|c|c|c|}
	\hline
	شماره & نام & پیشنیاز/همنیاز & 	تعداد واحد \\
	\hline
	\multicolumn{4}{|c|}{الف- سبد رباتیک} \\
	\hline
	28569 & اندازه گیری و سیستمهای کنترل & 28262  & 	2 \\
	\hline
	28615 & آز اندازه گیری و سیستمهای کنترل & 28569 هم نیاز & 	1 \\
	\hline
	28864 & رباتیک & 28567 & 	3 \\
	\hline
	28231 & آز رباتیک & 28864 هم نیاز & 	1 \\
	\hline
	\multicolumn{4}{|c|}{ب- سبد تبرید-تهویه} \\
	\hline
	28173 & سیستمهای تبرید & 28113 & 	3 \\
	\hline
	28167 & تهویه مطبوع & 28113 & 	3 \\
	\hline
	\multicolumn{4}{|c|}{ج- سبد مکانیک سیالات محاسباتی} \\
	\hline
	28439 & مقدمات سیالات محاسباتی & 28462 و 28113 & 	3 \\
	\hline
	28016 & دینامیک سیالات محاسباتی & 28439 & 	3 \\
	\hline
\end{tabular}
\caption{\label{mech-t2}
جدول شماره دو مکانیک: دروس سبدهای انتخابی
}
\end{center}
\end{table}


\end{document}