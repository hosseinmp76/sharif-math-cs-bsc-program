\documentclass{article}
\usepackage[utf8]{inputenc}
\usepackage[subpreambles=true]{standalone}
\usepackage{import}
\usepackage{amsthm}
\usepackage{pgffor}
\usepackage{latexsym}
\usepackage{xargs}
\usepackage{xparse}
\usepackage{tabularx}
\usepackage{etoolbox}
\usepackage{amssymb}
\usepackage{verbatim}
\usepackage{float}
\usepackage{enumitem,amsmath,array}
\usepackage{tikz}
\usepackage{tkz-graph}
\usepackage{bookmark}
\usetikzlibrary{positioning,chains,fit,shapes,calc}
\usepackage[a4paper, margin=1.0in]{geometry}
\usepackage{listings}
\usepackage{pgf}
\usepackage{pgfpages}
\usepackage{hyperref}
\usepackage{xifthen}% provides \isempty test
\usepackage{ragged2e}
\usepackage{graphicx}
\usepackage{blindtext}
\usepackage{mathrsfs}

% add your packeges here.
% this is added for s002
\usepackage{mathtools}
% this is added for s002
\usepackage{subcaption}
% this is added for s002

\graphicspath{{images/}}
\hypersetup{
    colorlinks=true,
    linkcolor=blue,
    filecolor=magenta,
    urlcolor=cyan,
}

% below packages have come from here :https://tex.stackexchange.com/questions/287131/which-encoding-used-in-latex-sources-to-convert-pdf-file-accents-to-the-doc-form
\usepackage{lmodern}
\usepackage[T1]{fontenc}



\usepackage{background}
\usetikzlibrary{calc}
\backgroundsetup{angle = 0, scale = 1, vshift = -2ex,
	contents = {\tikz[overlay, remember picture]
		\draw [rounded corners = 20pt, line width = 1pt,
		color = blue, double = black!10]
		($(current page.north west)+(-7cm,11cm)$)
		rectangle ($(current page.south east)+(-9cm,15cm)$);
	}
	}
\pagestyle{empty}
\usepackage{import}
\usepackage{amsthm}
\usepackage{latexsym}
\usepackage{tabularx}
\usepackage{amssymb}
\usepackage{verbatim}
\usepackage{float}
\usepackage{enumitem,amsmath,array}
\usepackage{tikz}
\usepackage{tkz-graph}
\usepackage{bookmark}
\usetikzlibrary{positioning,chains,fit,shapes,calc}
\usepackage[a4paper, margin=1.0in]{geometry}
\usepackage{listings}
\usepackage{pgf}
\usepackage{pgfpages}
\usepackage{hyperref}
\usepackage{xifthen}% provides \isempty test
\usepackage{ragged2e}
\usepackage{graphicx}
\usepackage{blindtext}
\usepackage{mathrsfs}


\newcommand{\loopy}[6]{
    \fbox{\parbox{\textwidth}{
    		\hypertarget{#1}{
        نام درس: #1
    }
        \\
        شمارهی درس: #2
        \\
        تعداد واحد: #3
        \\پیشنیاز: #4
    }}
    \\
    \ifthenelse{\isempty{#5}}
    {}{
        \fbox{\parbox{\textwidth}{
            اهداف آموزشی:
            \begin{enumerate}
                \foreach \x in #5 {
                    \item \x
                }
            \end{enumerate}
        }}
    }
    \ifthenelse{\isempty{#6}}
    {}{
        \fbox{\parbox{\textwidth}{
            ریز مواد:
            \begin{enumerate}
                \foreach \x in #6 {
                    \item \x
                }
            \end{enumerate}
        }
        }
    }
}

\begingroup
\catcode`@=11
\long\gdef\firstofmany#1{%
	\@fom{\unexpanded{[#1]}}#1{[#1]}}

\long\gdef\@fom#1#2{%
	\unexpanded{#2}%
	\@gobbleto{#1}}

\gdef\@gobbleto#1#2{%
	\ifnum\pdfstrcmp{\unexpanded{#2}}{#1}=\z@
	\expandafter\@gobbletwo
	\else
	\fi
	\@gobbleto{#1}}

\gdef\@gobbletwo#1#2{}

\endgroup


% Example usage
\newcommand{\Cone}[6]{%
	% Code for C1 command using the 6 arguments #1, #2, #3, #4, #5, #6
	% ...
	% ...
	% ...
	\pagebreak
	\loopy{#1}{#2}{#3}{#4}{#5}{#6}
	
}

\newcommand{\callConeList}[1]{
  	\begin{enumerate}
		\renewcommand*{\do}[1]{%
			\item \hyperlink{\firstofmany{##1}}{\firstofmany{##1}}
		}
		\docsvlist{#1}% Process the list to create an enumerate environment
	\end{enumerate}
    \renewcommand*{\do}[1]{\Cone##1}% Call C1 with 6 arguments from each element of the list
    \docsvlist{#1}% Process the list
}



%\usepackage[fontsloadable]{xepersian}
\usepackage{xepersian}

%\settextfont[
%Scale=1.09,
%Extension=.ttf,
%Path=./font/,
%ItalicFont=*It,
%BoldFont=*Bd,
%BoldItalicFont=*BdIt,
%]{HM XNiloofar}

\settextfont[
Scale=1.09,
Extension=.ttf,
Path=./font/,
%ItalicFont=*It,
BoldFont=Vazirmatn-Bold,
%BoldItalicFont=*BdIt,
]{Vazirmatn-Regular}

\title{
برنامه‌ی مقطع کارشناسی رشته‌ی ریاضیات و کاربردها
}
\author{
دانشکده علوم ریاضی
}
\date{
    تیر 1402
}


\begin{document}
    \maketitle
    \section{مقدمه }
    این برنامه بر اساس آخرین برنامه‌ی مصوب رشته‌های علوم ریاضی در وزارت علوم، تحقیقات و فناوری، و با توجه به شرایط فعلی دانشکده‌ی علوم ریاضی و رشته‌ی ریاضیات و کاربردها تدوین شده است. بدیهی است برنامه می‌تواند در بخش‌هایی که در اختیار دانشکده است، بر حسب نیازهای جاری و آتی تغییر کند و مجدداً جهت اجرا به دانشجویان ابلاغ شود. این برنامه برای دانشجویان ورودی 1401 و ما بعد لازم‌الاجرا و برای دانشجویان 1400 و ما قبل پس از تأیید استاد راهنما و دانشکده، قابل اجرا است. این برنامه در دو گرایش «محض» و «کاربردی» تنظیم شده است.
	\pagebreak
	\tableofcontents
	\pagebreak
    \listoftables
    \newpage

    \section{ساختار کلّی برنامه}
    \subsection{
    گرایش محض
}
    برنامه شامل بخشهای زیر است:
    \begin{table}[H]
    \begin{center}
        \begin{tabular}{|c | c | c|}
            \hline
            {دروس عمومی} & & {20 واحد} \\
            \hline
            {
            	دروس الزامی دانشگاه (پایه)
            	}
            & {
   			\href{math-t1}{
   			جدول 
   			}
            	 } & {23 واحد} \\
            \hline
            {دروس الزامی دانشکده} & {
            		\href{math-t2}{
            		جدول 
            		}
            } & {16 واحد} \\
            \hline
            {دروس الزامی-تخصصی} & {
            		\href{math-t3-1}{
            		جدول 
            		}
            } & {20 واحد} \\
            \hline
            {دروس انتخابی-تخصصی} & & {39 واحد} \\
            \hline
            {دروس اختیاری(خارج دانشکده) } & & {17 واحد} \\
            \hline
            {جمع کل واحدها} & & {135 واحد} \\
            \hline
        \end{tabular}
		\caption{\label{math-t1}
	ساختار کلی برنامه گرایش محض
		}
		\end{center}
		\end{table}
    \subsection{
	گرایش کاربردی
}
برنامه شامل بخشهای زیر است:
    \begin{table}[H]
\begin{center}
	\begin{tabular}{|c | c | c|}
		\hline
		{دروس عمومی} & & {20 واحد} \\
		\hline
		{دروس الزامی دانشگاه (پایه)}\LR{)} & {
		\href{math-t1}{
		جدول
		}
		} & {23 واحد} \\
		\hline
		{دروس الزامی دانشکده} & {
			\href{math-t2}{
			جدول 
			}
			
			} & {16 واحد} \\
		\hline
		{دروس الزامی-تخصصی} & {
				\href{math-t3-1}{
				جدول 
				}
		} & {24 واحد} \\
		\hline
		{دروس انتخابی-تخصصی} & & {35 واحد} \\
		\hline
		{دروس اختیاری(خارج دانشکده) } & & {17 واحد} \\
		\hline
		{جمع کل واحدها} & & {135 واحد} \\
		\hline
	\end{tabular}
    \end{center}
    \caption{\label{math-t3-2}
	ساختار کلی برنامه گرایش کاربردی
	   }
    \end{table}
    \section{مقررات برنامه}
    \begin{enumerate}
        \item
        مقررات اخذ 20 واحد دروس عمومی تابع ضوابط تعیین شده توسط معاونت آموزشی و تحصیلات تکمیلی در دانشگاه است و دانشجو ملزم به اخذ و گذراندن دروس مربوطه در چارچوب ضوابط تعیین شده خواهد بود.
        \item
        اخذ و گذراندن تمامی دروس جدولهای
        \ref{math-t1}
        ،
		\ref{math-t2}
		و همچنین
		\ref{math-t3-1}
		یا
		\ref{math-t3-2}
         توسط دانشجو الزامی است.
        \item
		دانشجویان دوره کارشناسی ریاضیات و کاربردها با گرایش محض باید 39 واحد از دروس «انتخابی-تخصصی» را با رعایت مقررات زیر بگذرانند:
        \begin{enumerate}
            \item
          دروس انتخابی-تخصصی به 8 زمینه موضوعی به شرح زیر تقسیم شده‌اند که دانشجو باید 30 واحد از 5 یا 6 زمینه، و در هر زمینه حداکثر 8 واحد، درس بگذراند:
            \begin{enumerate}
                 \item
					آمار و احتمال
                 \item
						آنالیز
                 \item
					آنالیز عددی و بهینه‌سازی
                 \item
					ترکیبیات
                 \item
					جبر و نظریه اعداد
                 \item
					معادلات دیفرانسیل و سیستم‌های دینامیکی
                 \item
					منطق و نظریه مجموعه‌ها
                 \item
					هندسه و توپولوژی
            \end{enumerate}
            \item
		زمینه‌ تخصصی دروس انتخابی-تخصصی، از بین زمینه‌های تخصصی بالا، توسط دانشکده علوم ریاضی تعیین و اعلام می‌شود.
            \item
             مشروط بر اینکه دانشجو شرایط و ضوابط لازم دانشکده برای ثبت‌نام در دروس دوره تحصیلات تکمیلی دانشکده علوم ریاضی را دارا باشد، علاوه بر دروس مقطع کارشناسی، دروس گذرانده شده دوره تحصیلات تکمیلی در زمینه مربوط نیز با موافقت پیشین استاد راهنما و تایید مسئول تطبیق، قابل تطبیق است.
             \item
             سایر واحدهای باقیمانده در بخش «دروس انتخابی-تخصصی می‌تواند با گذراندن هر یک از دروس ارائه شده در دانشکده علوم ریاضی (از جمله هر یک از دروس مقطع کارشناسی رشته‌های علوم کامپیوتر و ریاضیات و کاربردها، دروس مقطع تحصیلات تکمیلی، سمینار یا پروژه کار‌شناسی) با رعایت مقررات آموزشی و تایید استاد راهنما تکمیل شود.
        \end{enumerate}
        \item
        دانشجویان دوره کارشناسی ریاضیات و کاربردها با گرایش کاربردی باید 35 واحد از دروس «انتخابی-تخصصی» را با رعایت مقررات زیر بگذرانند:
        \begin{enumerate}
        	\item
        انتخاب حداقل چهار زمینه تخصصی متفاوت از زمینه‌های زیر و گذراندن حداقل 3 واحد در هر یک از این چهار زمینه‌ برای دانش‌آموختگی در این برنامه الزامی است. زمینه‌ تخصصی دروس انتخابی-تخصصی، از بین زمینه‌های تخصصی زیر، توسط دانشکده علوم ریاضی تعیین و اعلام می‌شود. زمینه‌های تخصصی دروس این برنامه در تاریخ تصویب شامل موارد زیر است
        \begin{enumerate}
			\item
				بهینه‌سازی و محاسبات علمی
        	\item
				آمار و علوم داده
        	\item
				معادلات دیفرانسیل و سیستم‌های دینامیکی
        	\item
				ریاضیات گسسته و ترکیبیاتی
        	\item
				ریاضیات تصادفی
        	\item
				هوش مصنوعی
        	\item
				ریاضیات زیستی
        	\item
				کدگذاری و رمزنگاری
			\end{enumerate}
    \item
زمینه‌ تخصصی دروس انتخابی-تخصصی، از بین زمینه‌های تخصصی بالا، توسط دانشکده علوم ریاضی تعیین و اعلام می‌شود.
	\item
		سایر واحدهای باقیمانده در بخش «دروس انتخابی-تخصصی» می‌تواند با گذراندن هر یک از دروس ارائه شده در دانشکده علوم ریاضی (از جمله هر یک از دروس مقطع کارشناسی رشته‌های علوم کامپیوتر و ریاضیات و کاربردها، دروس مقطع تحصیلات تکمیلی، سمینار یا پروژه کار‌شناسی) با رعایت مقررات آموزشی و تایید استاد راهنما تکمیل شود.
	\end{enumerate}
        \item
         «دروس اختیاری» شامل 17 واحد است و باید با موافقت استاد راهنما و با در نظر گرفتن تبصره‌های زیر انتخاب شوند:
         \begin{enumerate}
         	\item
		 		حداقل 12 واحد از این دروس الزاما باید از دروس خارج از دانشکده علوم ریاضی اخذ شوند.
         	\item
				برای دانشجویان گرایش کاربردی گذراندن حداقل 3 واحد در زمینه مدیریت یا اقتصاد با تایید استاد راهنما الزامی است.
         	\item
				درچارچوب قوانین و مقررات آموزشی دانشگاه، گذراندن حداکثر شش واحد از دروس مراکز (معارف، زبانها، کارگاه‌ها) و گروه‌ها (فلسفه علم) مجاز است. سایر واحدهای خارج از دانشکده باید از دروس تخصصی دانشکده‌های دیگر اخذ شود.
			\end{enumerate}
        \item
         \item
			سرفصل تمامی دروس، مطابق سرفصل درج شده در این برنامه یا بر اساس آخرین سرفصل اعلامی توسط دانشکده علوم ریاضی در مورد هر درس تعیین می‌شود.
         \item
			دانش‌آموختگی در این برنامه با شرکت در یک برنامه کهاد در چارچوب مقررات مندرج در بخش 
			\autoref{mathkehad}
			 این برنامه نیز برای هر دانشجو امکان‌پذیر است (الزامی نیست). برای شرکت هر دانشجو در یک برنامه کهاد، تایید دانشکده علوم ریاضی و دانشکده مقصد (یعنی دانشکده ارائه دهنده دوره کهاد مصوب برای رشته ریاضی) الزامی است. مجموع واحدهای دانشجویی که برنامه دوره خود را با یک کهاد مصوب به پایان می‌رساند می‌تواند تا سقف 138 واحد افزایش داشته باشد.
         \item
			مقررات اخذ دروس مازاد بر مجموع واحدهای دوره، مطابق قوانین ابلاغی از طرف اداره کل آموزش دانشگاه مشخص می‌شود.

    \end{enumerate}

\pagebreak

	\section{
	جداول دروس رشته
}
شماره های دروس، مطابق شمارهی دروس مربوطه در سیستم آموزش دانشگاه است.
\begin{table}[H]
\begin{center}

	\begin{tabular}{|c|c|c|c|}
		\hline
		شماره & نام & پیشنیاز/همنیاز & تعداد واحد  \\
		\hline
		22015 & ریاضیات عمومی 1 &  & 4 \\
		\hline
		22016 & ریاضیات عمومی 2 & 22015 & 4 \\
		\hline
		22034 & معادلات دیفرانسیل & همنیاز با 22016 & 3 \\
		\hline
		22048,22049 & مبانی کامپیوتر و برنامهنویسی &  & *3+1 \\
		\hline
		24011 & فیزیک 1 &  & 3 \\
		\hline
		24001 & آز فیزیک 1 & همنیاز با 24011 &  \\
		\hline
		24012 & فیزیک 2 & 24011 &  \\
		\hline
		33018 & کارگاه عمومی &  & 1 \\
		\hline
	\end{tabular}
    \caption{\label{math-t1}
دروس الزامی دانشگاه (پایه)
    }
    \end{center}
    \end{table}
* این درس بصورت سه واحد نظری و یک واحد عملی ارائه می شود.
\begin{table}[H]
\begin{center}
\begin{tabular}{|c|c|c|c|}
	\hline
	شماره & نام & پیشنیاز/همنیاز & تعداد واحد \\
	\hline
	22255 & جبر خطی 1 & 22016 & 4 \\
	\hline
	22089 & احتمال و کاربرد آن & 22016 & 4 \\
	\hline
	22655 & آنالیز عددی 1 & 22016 & 4 \\
	\hline
	22142 & مبانی ریاضیات &  & 4 \\
	\hline
    \end{tabular}
    \caption{\label{math-t2}
دروس الزامی دانشکده
    }
    \end{center}
    \end{table}
    \begin{table}[H]
    	\begin{center}
\begin{tabular}{|c|c|c|c|}
	\hline
	شماره & نام & پیش‌نیاز/هم‌نیاز & تعداد واحد \\
	\hline
	22325 & آنالیز ریاضی 1 &   & 4 \\
	\hline
	22326 & آنالیز ریاضی 2 & 22325 & 4 \\
	\hline
	22217 & جبر 1 &  & 4 \\
	\hline
	22556 & توپولوژی 1 & 22325 & 4 \\
	\hline
	22335 & توابع مختلط 1 & 22325 & 4 \\
	\hline
\end{tabular}
        \caption{\label{math-t3-1}
دروس الزامی-تخصصی گرایش محض
    }
    \end{center}
\end{table}
\begin{table}[H]
\begin{center}
\begin{tabular}{|c|c|c|c|}
	\hline
	شماره & نام & تعداد واحد & پیش‌نیاز/هم‌نیاز \\
	\hline
	22325 & آنالیز ریاضی 1 & 4 &  \\
	\hline
	22217 & جبر 1 & 4 &  \\
	\hline
	22825 & ریاضیات گسسته & 3 &  \\
	\hline
	22067 & آمار و کاربرد & 3 &  \\
	\hline
	C3 & برنامه‌نویسی پیشرفته & 3 & 22049 \\
	\hline
	22499 & بهینه‌سازی خطی  & 4 & 22255 \\
	\hline
	22035 & ریاضی مهندسی* & 3 & 22034 \\
        \hline
    \end{tabular}
    \caption{\label{math-t3-2}
     دروس الزامی-تخصصی گرایش کاربردی
    }
    \end{center}
    \end{table}
* دانشجویان می توانند به جای درس ریاضی مهندسی یکی از دروس توابع مختلط 1 (22335) یا معادلات دیفرانسیل با مشتقات جزئی (22395) را بگذرانند که در این صورت، یک واحد مازاد در جدول دروس تخصصی انتخابی قابل تطبیق است. گذراندن هر سه درس مذکور مجاز نیست.

\pagebreak

\section{برنامه کهاد}
\label{mathkehad}
به منظور افزایش و تشویق آموزش‌های بین‌رشته‌ای و در راستای برنامه‌ها و توصیه‌های وزارت عتف در برگزاری دوره‌های کهاد و به استناد برنامه دوره کارشناسی علوم ریاضی مصوب 26/2/1388، امکان انتخاب یک برنامه کهاد (شامل حداقل 24 و حداکثر 27 واحد درسی) توسط دانشجویان از بین برنامه‌های کهاد مصوب در دانشگاه برای رشته ریاضیات و کاربردها وجود دارد. در اینصورت دانشجوی  رشته ریاضیات و کاربردها با گذراندن یک بسته آموزشی کهاد می‌تواند با عنوان مدرک کارشناسی در رشته ریاضیات و کاربردها و کهاد مربوطه دانش‌آموخته شود. بدین منظور پس از تصویب و دریافت مجوز ادامه تحصیل در یک کهاد مصوب برای یک دانشجو از دانشکده‌های مبدا و مقصد، مجموعه دروس کهاد مصوب مربوطه در قالب «دروس اختیاری» و نیز بخشی از «دروس انتخابی-تخصصی» قابل تطبیق است. سقف تعداد واحدهای قابل تطبیق در قالب این جداول 24 واحد است که با اولویت ابتدا در جدول «دروس اختیاری» و پس از آن جدول «دروس انتخابی-تخصصی» انجام خواهد شد و واحدهای مازاد بر 24 واحد دوره کهاد باعث افزایش تعداد واحدهای لازم برای فارغ‌التحصیلی تا سقف 138 واحد خواهد شد و واحدهای افزایش‌یافته مازاد محسوب نخواهد شد. در تطبیق دروس دوره کهاد برای گرایش محض، رعایت بند 3-الف مربوط به جدول «دروس اختیاری» (بخش 3) الزامی است. همچنین، برای گرایش کاربردی رعایت بند 4-الف مربوط به جدول «دروس انتخابی-تخصصی» و تبصره 2 بند 5 مربوط به جدول «دروس اختیاری» الزامی است.

\subsection{
شرایط و مقررات عمومی کهاد
}
\begin{enumerate}
    \item
     دوره کهاد یک دوره اختیاری در برنامه رشته ریاضیات و کاربردها است که به جهت ایجاد فضای میان رشته ای و آموزش دانش آموختگانی با علائق و تواناییهای متنوع و ترکیبی طراحی شده است. لذا، علاقه دانشجو به رشته کهاد مورد نظر و دقت در انتخاب آن و همچنین بهره گیری از مشاوره لازم در طی دوره توسط دانشجو از ضروریات موفقیت در این دوره است. دوره کهاد یک امکان و انتخاب در برنامه با اهداف خاص است و پذیرش دانشجو در این دوره ها اساسا به صورت بسیار محدود و با دقت نظر لازم توسط دانشکده علوم ریاضی و دانشکده مقصد (دانشکده ارائه دهنده دوره کهاد) صورت می پذیرد.
	\item
	 درخواست دانشجو برای شرکت در برنامه کهاد باید پس از پایان نیمسال چهارم ارائه شده و حداکثر تا قبل از شروع نیمسال ششم تحصیلی تعیین تکلیف و نتیجه درخواست نهایی شود.
	\item
	 برای تایید درخواست دانشجو مبنی بر شرکت و ادامه تحصیل در دوره کهاد در برنامه رشته ریاضیات و کاربردها، موافقت دانشکده علوم ریاضی، موافقت دانشکده مقصد (یعنی دانشکده ارائه دهنده دوره کهاد) و رعایت کلیه مقررات و آیین‌نامه‌های دانشکده مقصد و اداره کل آموزش دانشگاه در ارتباط با دوره کهاد مربوطه الزامی است.
	\item
	 استاد راهنمای دوره کهاد معاون آموزشی دانشکده مقصد یا یک نفر از اساتیدی است که توسط دانشکده مقصد تعیین می شود و دانشجو برنامه کهاد خود را زیر نظر ایشان و با رعایت کلیه مقررات دانشکده مقصد و دانشکده علوم ریاضی دنبال خواهد کرد. مشاوره با معاون آموزشی یا اساتید متخصص در دانشکده مقصد قبل از ارائه درخواست توسط دانشجو و همچنین در طی دوره موکدا توصیه می شود.
	\item
	 برنامه‌های کهاد مصوب دانشگاه در رشته ریاضیات و کاربردها اساسا با تایید و نظارت معاونت آموزشی دانشگاه اجرا می شود و اخذ کلیه مجوزهای لازم در هر مرحله و رعایت کلیه مقررات جاری اداره کل آموزش دانشگاه در ارتباط با ادامه تحصیل در دوره کهاد مربوطه الزامی است.
	\item
	 با توجه به مقررات و در صورت وجود مجوزهای لازم، امکان درج نام کهاد مربوطه در دانشنامه پایان تحصیلات مقطع کارشناسی وجود خواهد داشت و پس از پایان موفقیت‌آمیز دوره کارشناسی ریاضیات و کاربردها، دانشجویی که رشته ریاضیات و کاربردها را با یک کهاد مصوب به پایان برساند می‌تواند در رشته ریاضیات و کاربردها و با درج نام کهاد مربوطه در دانشنامه خود دانش‌آموخته شود.
	\item
	رعایت کلیه مقررات آموزشی دانشگاه و وزارت علوم، تحقیقات و فناوری، از جمله سقف سنوات تحصیلی، سقف و کف واحدهای مجاز در هر نیمسال، رعایت پیش‌نیازی و هم‌نیازی دروس و ضوابط مشروطی در طول دوره تحصیل دانشجو الزامی است و دانشجوی پذیرفته شده در یک دوره کهاد همانند سایر دانشجویان ملزم به رعایت تمامی مقررات مرتبط با دوره تحصیلی خود خواهد بود.
\end{enumerate}

\section{برنامه کهادهای مصوب}
کهادهای زیر در زمان تصویب این برنامه با هماهنگی و تایید دانشکده‌های ذیربط و تایید دانشگاه تعیین و تصویب شده اند و از تاریخ تصویب این برنامه با رعایت تمامی مقررات مربوطه قابل اجرا هستند. دانشکده علوم ریاضی می تواند با هماهنگی با دانشکده‌های دیگر و تایید دانشگاه، نسبت به بازنگری برنامه های کهاد موجود یا اضافه کردن برنامه های کهاد جدید اقدام و پس از تصویب در دانشگاه به این مجموعه جهت اجرا اضافه کند.
این دوره‌ها به‌عنوان يك دوره آموزشي اختياري براي دانشجويان رشته ریاضیات و کاربردها در دانشكده علوم ریاضی كه علاقه‌مند به آشنایی و فراگیری ابعادی از رشته کهاد مورد نظر که مرتبط با و یا مکمل رشته تحصیلی خود ‌باشند طراحی‌شده است. دوره‌های کهاد خصوصاً براي دانشجویانی كه قصد دارند در زمینه‌هاي بین‌رشته‌ای و نوين علم و فناوري ادامه فعالیت دهند، بسيار مناسب است.



\subsection{
کهاد مهندسی مکانیک
}
دوره کهاد مهندسي مكانيك شامل 24 واحد درسی است که 18 واحد آن الزاما باید از بین مجموعه دروس اصلی-الزامی این کهاد (
\href{mech-t1}{جدول اول مکانیک}
) و 6 واحد باقیمانده منطبق با اهداف نهايي دانشجو، صرفا از بین یکی از سه سبد انتخابی این کهاد (
\href{mech-t2}{جدول دوم مکانیک}
) انتخاب شود.
\subsection{
کهاد اقتصاد
}
دوره کهاد اقتصاد شامل 24 واحد درسی است که  12 واحد آن باید مطابق دروس اصلی-الزامی کهاد(
\href{eco-t1}{جدول اول اقتصاد}
) و 6 واحد آن از بین مجموعه دروس انتخابی-الزامی این دوره (
 \href{eco-t2}{جدول دوم اقتصاد}
 ) اخذ شوند. شش واحد باقیمانده می تواند از مابقی دروس انتخابی-الزامی (
  \href{eco-t2}{جدول دوم اقتصاد}
 )(که قبلا به عنوان درس انتخابی تطبیق نشده باشند) یا هر یک از دروس ج
  \href{eco-t3}{جدول سوم اقتصاد}
 انتخاب شوند. برای دانشجویانی که در این کهاد پذیرفته می شوند، یکی از دروس دوره کهاد به عنوان درس مورد نیاز برای ارضای تبصره 2 بند 4 در بخش 3 (گذراندن حداقل 3 واحد در زمینه مدیریت یا اقتصاد) تطبیق خواهد شد.
\subsection{
کهاد مهندسی صنایع
}
دوره کهاد مهندسي صنایع شامل 24 واحد درسی است که 18 واحد آن الزاما باید از بین مجموعه دروس اصلی-الزامی این کهاد (
\href{ind-t1}{جدول اول صنایع}
) و 6 واحد باقیمانده منطبق با اهداف نهايي دانشجو، صرفا از بین یکی از سه سبد انتخابی این کهاد (
\href{ind-t2}{جدول دوم صنایع}
) انتخاب شود.
\subsection{
کهاد علوم کامپیوتر
}
دوره کهاد علوم کامپیوتر شامل 24 واحد درسی است که 9 واحد آن شامل دروس اصول سیستم‌های کامپیوتری  (C4)، ساختمان داده‌ها (22815) و نظریه‌ی زبان‌ها و اتوماتا (22825) است و 15 واحد باقیمانده، باید از بین سایر دروس الزامی-تخصصی یا انتخابی-تخصصی رشته علوم کامپیوتر (که دانشجو در رشته خود نگذرانیده است) انتخاب شود.

\pagebreak
\section{جداول دروس برنامه های کهاد}

\import{./}{mechanics_minor}
\import{./}{economics_minor}
\import{./}{industrial_minor}
\import{./}{courses}
 \end{document}