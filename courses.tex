\documentclass[class=article, crop=false]{standalone}
% Document
\begin{document}
    \section{
        ریز مواد دروس
    }
    \callConeList{
            {{
            ریاضی عمومی 1/ریاضی عمومی 2
        }{
            22015/22016
        }{
            4/4
        }{
            ندارد /ریاضی عمومی 1
        }{{
            آشنا ساختن دانشجویان با حساب دیفرانسیل و انتگرال به عنوان ابزار حل مسایل بالاخص مسائل غیر خطی.
            ,
            معرفی مفاهیم جبر خطی در ابعاد بالا به عنوان زمینه طرح و بررسی مسایل بعد بالا.
            ,
            کمک به درک مفهوم اصل تقریب و ایجاد انگیزه های محاسباتی برای حل مسائل با استفاده از ابزار ماشین حساب و کامپیوتر.
            ,
            تاکید بر مفاهیم و شهود اجتناب از تکیه بر روشها و تکنیکهای محاسباتی که امروزه به کمک ماشین حساب و کامپیوتر به سادگی انجام می شود.
            ,
            در عین تاکید بر مفاهیم اصلی ریاضی از تجرید بی انگیزه قویاً اجتناب شود. هدف این درسها فراهم آوردن چهارچوب مفهومی مناسب و ابزار ضروری برای صورتبندی مسایل به صورت ریاضی و حل آنهاست.
            ,
            مفهوم معادلات دیفرانسیل و دستگاه معادلات دیفرانسیل در سراسر درسها به طور طبیعی ظاهر می شود. مسایل رشد و زوال حرکتهای نوسانی و سایر پدیده های تحولی خطی و غیر خطی در رابطه با معادلات دیفرانسیل مطرح شود.
            ,
            چینش مطالب بر اساس اهداف آموزشی گذاشته شود تا سلسله مراتب موضوعی ارائه مطالب به صورتی باشد که دانشجویان احساس تکراری بودن آن را نسبت به برنامه دبیرستان نکنند.
            ,
            با توجه به اینکه این دو درس پیشنیاز درسهای معادلات دیفرانسیل و ریاضی مهندسی هستند مطالبی که مطرح کردن آنها در این دو درس مناسب و به کم حجم شدن درسهای معادلات دیفرانسیل و ریاضی مهندسی کمک می کند در این دو درس مطرح گردند.
            ,
            ریز مواد ریاضی عمومی 1 و 2 به صورت یک درس یک ساله نوشته شود تا امکان انعطاف تدریس این دو درس در سالهای مختلف فراهم گردد.
            ,
        }}{{
            اعداد: مروری تاریخی بر مفهوم عدد اعداد گویا و ناگویا، اصل تمامیت، اعداد مختلط و برخی کاربردهای آنها دنباله ها و سریهای عددی.
            ,
            توابع یک متغیری: حد و پیوستگی خواص تابعهای پیوسته روی یک بازهٔ بسته مشتق پذیری، تقریب خطی، کاربردهای مشتق، چند جمله ای تیلور و کاربردهای آن.
            ,
            انتگرال یک متغیری: انتگرالهای معین و نامعین، قضایای اساسی، تابعهای متعالی، معادلات دیفرانسیل، روشهای تقریب، کاربردهای سنتی انتگرال منجمله مختصری در مورد احتمال.
            ,
            معادلات دیفرانسیل: مسایل رشد و زوال، حرکتهای نوسانی.
            ,
            سریهای تابعی: سریهای توانی، سری تیلور، و سری فوریه، کاربردها، منجمله حل معادلات دیفرانسیل به وسیلهٔ سریهای توانی
            ,
            معرفی مفهوم جبر خطی n بعدی: خواص خطی Rn ضرب داخلی و کاربردهای آن، زیر فضاها، تابعهای خطی و کاربرد آنها، مفاهیم حجم دترمینان، قطری کردن ماتریسهای متقارن.
            ,
            خمها در صفحه و فضا: مفاهیم انحنا و تاب و قضایای اساسی.
            ,
            توابع Rn به Rn: خواص عمومی، نمایش توابع چند متغیری، مفاهیم حد، پیوستگی و مشتقهای جزئی.
            ,
            مشتق توابع چند متغیری: مشتق پذیری، گرادیان، قاعده زنجیره ای، مشتقات مرتبهٔ بالا، چند جمله ای و سری تیلور چند متغیره، قضایای تابع معکوس و تابع ضمنی.
            ,
            بهینه سازی: نقاط بحرانی و عادی، رده بندی نقاط بحرانی، یافتن ماکسیمم و مینیمم بدون قید و با قید روش لاگرانژ.
            ,
            انتگرال چند گانه: مفاهیم اصلی، محاسبه، انتگرالهای ناسره، فرمول عمومی تعویض متغیر.
            ,
            انتگرال روی خم و میدانهای برداری: مفاهیم اصلی و کاربرد، محاسبه، میدانهای پایسته و پتانسیل.
            ,
            انتگرال روی سطوح خمیده: بررسی رویه های هموار پارامتری و عمومی، انتگرال روی سطح و کاربردهای آن.
            ,
            آنالیز برداری: مفاهیم دیورژانس و کرل و تعبیر هندسی و فیزیکی آنها، قضایای گرین، استوکس و دیورژانس به صورتهای مختلف، کاربرد در مسایل پتانسیل اسکالر و برداری.
            ,
        }}
        },{{
            معادلات دیفرانسیل
        }{
            22034
        }{
            3
        }{
            ریاضی عمومی 2 یا همزمان
        }{{
            1- تاکید بر مدلسازی و مطالعه مدلهای ریاضی سیستمهای فیزیکی، طبیعی و اجتماعی. 2- مطالعه معادلات دیفرانسیل با روشهای تحلیلی، هندسی و کیفی. 3- تاکید بر مفاهیم و شهود و اجتناب از تکیه بر روشها و تکنیکهای محاسباتی که امروزه به کمک ماشین حساب و کامپیوتر به سادگی انجام می شود. 4- استفاده از نرم افزارهای ریاضی برای حل معادلات دیفرانسیل.
        }}{{
            حل معادلات دیفرانسیل عادی به وسیله روشهای تحلیلی، هندسی، و کیفی، معادلات دیفرانسیل عادی خطی بخصوص درجه دوم، استقلال خطی جوابها، روش ضرایب نامعین و تغییر پارامترها، سیستم معادلات خطی، روش ضرایب نامعین، معادلات غیر خطی خودگردان، نقطه های تکین، پایداری و پایداری مجانبی، روش دوم لیاپونف، مساله شکار و شکارچی، سری فوریه، معادلات دیفرانسیل با مشتقات جزئی مرتبه دوم، حرارت، موج، لاپلاس.
        }}
        },{{
            ریاضی مهندسی
        }{
            22035
        }{
            3
        }{
            معادلات دیفرانسیل
        }{{
            1- ارائه مباحث گسترده ای از توابع مختلط و معادلات دیفرانسیل. 2- اختصاص 50 درصد از درس به هر یک از این دو مبحث توابع مختلط و معادلات دیفرانسیل.
            3- یادگیری تکنیکهای محاسباتی، به کارگیری صورت قضیه ها در حل مسئله ها. 4- تاکید بر کاربرد قضایا. 5- تکیه بر یکی از دو مبحث نگاشتهای همدیس یا انتگرال مختلط، متناسب با نیازها.
        }}{{
            توابع مختلط، تحلیلی بودن، انتگرال روی خم قضیه انتگرال کشی، نقاط تکین، سری تیلور ولوران، مانده، محاسبه انتگرالهای حقیقی به وسیله مانده ها، نگاشتهای همدیس، تبدیل لاپلاس و فوریه، تابع دلتای دیراک و کاربرد آنها در حل معادلات دیفرانسیل عادی، توابع خاص و مسائل با شرایط مرزی، مسئله اشترم لیوویل، معادلات دیفرانسیل پارهای مرتبه دوم چند متغیره.
        }}
        },{{
            جبر خطی 1
        }{
            22255
        }{
            4
        }{
            ریاضی عمومی 2
        }{{
            1- تدریس جبر خطی 1 با تاکید روی R، C به گونه ای که در این درس مطالب مورد نیاز به عنوان پیشنیاز دروس مختلف ریاضی پوشانده شده باشد. 2- ایجاد توانایی در دانشجو جهت یادگیری، خواندن و ساختن اثباتها و همین طور محاسبات مربوط به موضوع درس بالاخص سطری پلکانی کردن ماتریسها، محاسبات مقادیر ویژه، به دست آوردن فرمهای ژردن، پیدا کردن پایه یک فضای برداری. 3- سعی در تفسیر هندسی مفاهیم. 4- تاکید بر تعامد و فضاهای ضرب داخلی.
        }}{{
            روشهای حذفی در حل معادلات خطی تجزیه LU، LDU فضای برداری و ریز فضاهای برداری، تبدیل خطی و ماتریس آن، معکوس ماتریس، ماتریسهای معکوس پذیر و خواص آن، پایه و بعد فضاهای برداری، مختصات و تعویض پایه، فضاهای پوچ و فضاهای ستونی یک ماتریس، دترمینان، کاربردهای دترمینان به خصوص تعبیر حجم، مقدارهای ویژه، بردارهای ویژه، فضاهای ویژه ماتریسهای مشابه، قضیه کلی – هامیلتون، قطری کردن، مثلثی کردن و فرمهای ژردن، فضای ضرب داخلی و تعامد، روش کوچکترین مربعات، ماتریسهای متعامد، متقارن و هرمیتی، ماتریسهای مثبت معین، قطری کردن ماتریسهای مثبت و معین.
        }}
        },{{
            آنالیز عددی 1
        }{
            22655
        }{
            4
        }{
            ریاضی عمومی 2
        }{{
            -1 طرح و تحلیل الگوریتمهای موثر برای حل مسایل علمی با تاکید بر شناسایی خصوصیاتی از قبیل حل مساله، پایداری، همگرایی و کارایی با الگوریتمها.
        }}{{
            نمایش ممیز شناور اعداد حقیقی و انواع مختلف خطاها، حالت مساله و پایداری الگوریتم، حل دستگاه معادلات خطی و تحلیل خطای محاسباتی، درونیابی، برازش داده به وسیله کمترین مربعات خطی، مساله نقطه ثابت و ارتباط با ریشه یابی توابع و مینیمم سازی، همگرایی و نرخ همگرایی روشهای تکراری نقطه ثابت، روش نیوتن برای حل دستگاه های غیر خطی و مینیمم سازی توابع چند متغیره، مشتقگیری عددی و مرتبه خطای برشی، انتگرالگیری عددی (روشهای نیوتن–کوتز، وفقی، رامبرگ، گوسی و انتگرالهای ناسره) حل معادلات دیفرانسیل عادی با شرایط اولیه (روشهای تک قدمه و چند قدمه.)
        }}
        },{{
            احتمال و کاربرد آن
        }{
            22089
        }{
            4
        }{
            ریاضی عمومی 2
        }{{
            1- تدریس احتمال با پیشنیاز ریاضی عمومی به گونه ای که در این درس مطالب مورد نیاز به عنوان پیشنیاز درسهای آماری و فرایندهای تصادفی، شبیه سازی و غیره پوشانده شود. 2- ایجاد توانایی در دانشجو جهت یادگیری و ساختن مدلهای ریاضی برای پدیده های تصادفی. 3- ایجاد توانایی در فهمیدن مفاهیم ریاضی مرتبط با موضوع درس و انجام محاسبات.
        }}{{
            فضای احتمال، جبر پیشامدها، مروری بر روشهای شمارش، احتمال شرطی و استقلال، متغیرهای تصادفی (واریانس و کوواریانس، گشتاورها و غیره)، متغیرهای تصادفی گسسته، توزیعهای متداول (دو جمله ای، هندسی، فوق هندسی، دو جمله ای منفی و پواسن)، دنباله های برنولی، فرایند پواسن، تقریب پواسن به وسیله چند جمله ای، متغیرهای تصادفی پیوسته، تابع چگالی احتمال، متغیرهای تصادفی پیوسته متداول، توزیعهای چند گانه، توزیع توام، توزیع نرمال چند متغیره، توزیع شرطی، امید شرطی، تابع مولد گشتاور، مجموع متغیرهای تصادفی مستقل، نامساوی چپیچف، قانون اعداد بزرگ، قضیه حد مرکزی.
        }}
        },{{
            آنالیز ریاضی 1
        }{
            22325
        }{
            4
        }{
            ندارد
        }{{
            1- تدریس آنالیز ریاضی توابع یک متغیره حقیقی به گونه ای که در این درس مطالب مورد نیاز به عنوان پیشنیاز دروس مختلف ریاضی پوشانده شده باشد. 2- ایجاد توانایی در دانشجو جهت یادگیری، خواندن و ساختن اثباتها.
        }}{{
            اعداد حقیقی، دنباله ها، حد زیرینه و زبرینه دنباله ها در R، مفهوم ابتدایی فضای متریک مانند فشردگی، همبندی، توابع پیوسته، توابع یکنوا، مشتق، قضیه میانگین، چند جمله ای تیلور، انتگرال ریمان و داربو در R، انتگرال پذیری، قضیه اساسی حساب دیفرانسیل و انتگرال، انتگرال ناسره متداول، همگرایی سری و فضای تابعی و همگرایی یکنواخت و قضایای تعویض حد، قضیه تقریب وایرشتراس، انتگرال و مشتق، سری توانی و تیلور و قضایای اساسی آنها، قضیه آبل.
        }}
        },{{
            آمار و کاربرد آن
        }{
            22064
        }{
            3
        }{
            احتمال و کاربرد آن
        }{{
            ایجاد توانایی در دانشجو جهت به کارگیری مفاهیم احتمال و روشهای آماری برای استخراج نتایج و انجام براوردهای آماری جهت استنتاج و نتیجه گیری در مورد جمعیتهای مورد مطالعه، آشنایی با روشهای گوناگون گرد آوری داده ها، آشنایی با روشهای گوناگون توصیف داده ها و ارائه نتایج آزمونهای آماری و آشنایی با نرم افزارهای جدید در این مورد و استفاده از آن.
        }}{{
            یادآوری توزیعهای احتمال مهم، آشنایی با آمار توصیفی، آماره ها، برآوردهای نقطهای و بازهای، آزمونهای فرض آماری، انواع خطاها، سطح تشخیص، توان آزمون، آزمونهای یکطرفه، آزمونهای دو طرفه، بازه های اطمینان، روشهای طراحی آزمونها و اجرای آنها، آزمونهای فرض میانگین با واریانس معلوم، آزمونهای فرض میانگین با واریانس نامعلوم، آزمونهای نسبت میانگینها، آزمونهای فرض واریانس، روشهای حداکثر احتمال، آزمون نکویی برازش، آشنایی با مدلهای رگرسیون و تحلیل واریانس، آشنایی با آمار غیر پارامتری
        }}
        },{{
            ریاضیات گسسته
        }{
            22825
        }{
            3
        }{
            ندارد
        }{}{{
            دوره سریع از مفاهیم مربوط به مجموعه ها، مجموعه توانی، تابع مشخصه و مفاهیم اولیه منطق پایه، انواع روابط روی مجموعه ها، آشنایی با مفاهیم اصلی و شمارش ضرایب چند جمله ای، روابط بازگشتی، توابع مولد، اصول شمول و عدم شمول، آشنایی با مربعهای لاتین و سیستم نمایندگی متمایز و ارتباط با هندسه های متناهی، آشنایی با مفاهیم و قضایای اصلی در نظریه گراف در حد مقدماتی از مفاهیم پایه شامل دور، مسیر، همبندی درجه و دنباله درجهای، انواع اصلی گرافها نظیر گرافهای کامل، دو بخشی و…. گرافهای اویلری و هامیلتونی، آشنایی با گرافهای جهت دار و تورنمنتها، آشنایی با مفاهیم تطابق کامل و ماکسیمم و قضایای اصلی در این مورد با تاکید بر الگوریتم پیدا کردن هر یک از آنها، آشنایی با مفاهیم اولیه در نظریه طرحها و ماتریسهای آدامار و ارتباط آنها با مفاهیم قبلی نظیر گرافها، مربعهای لاتین، هندسه های متناهی با تاکید بر مثال، آشنایی با مفهوم رنگآمیزی گراف و ارتباط آن با مفاهیم قبلی نظیر مربعهای لاتین و طرحها با تاکید بر مثال و همچنین چند جمله ای رنگی گرافها. تاکید درس بر کاربردها و روشهای الگوریتمی است.
        }}
        },{{
            آنالیز عددی 2
        }{
            22657
        }{
            4
        }{
            آنالیز عددی 1
        }{}{{
            محاسبه تجزیه های قائم ماتریسها، روشهای تکراری برای حل دستگاه های خطی، مسئله مقادیر ویژه و روشهای تکراری برای حل آن، محاسبه روشهای LR، QR مسئله مقادیر تکین و تجزیه مقادیر تکین، حل مساله کمترین مربعات با استفاده از تجزیه های قائم، حل معادلات دیفرانسیل عادی پارهای، روشهای تفاضلی و تقریبی، معادلات دیفرانسیلSTIFF همگرایی و نرخ همگرایی در روشهای تکراری.
        }}
        },{{
            جبر 1
        }{
            22217
        }{
            4
        }{
            ندارد
        }{{
            1- تدریس جبر 1 و ساختمانهای جبری مانند گروه، حلقه، میدان و ساختمانهای خارج قسمت و هم ریختیهای آن به گونه ای که در این درس مطالب مورد نیاز به عنوان پیشنیاز دروس مختلف ریاضی پوشانده شده باشد. 2- ایجاد توانایی دانشجو جهت یادگیری، خواندن و ساختن اثبات قضیه های درس. 3- ایجاد توانایی دانشجو جهت درک ساختمانهای مجرد جبری.ctives
        }}{{
            1- گروه ها: تعریف و مثالهای مهم چون گروه جایگشتها و گروه های دوری، زیر گروه و همدسته، قضیه کیلی، قضیه لاگرانژ، هم ریختی قضایا و خواص مربوط به آن، یکریختی گروه ها، حاصلضرب مستقیم گروه ها، مباحث مقدماتی در مورد گروه های بطور متناهی تولید شده. 2- حلقه و هیات: تعریف و مثالهای مهم، حوزه صحیح، هیات، زیر حلقه، ایده آل، حلقه خارج قسمت، هم ریختی و قضایا و خواص مربوط به آن، یکریختی حلقه ها، ایدهآلهای اول و ماکزیمال، مشخصه یک هیات و هیات اول، هیات کسرها، حلقه چند جمله ایها، الگوریتم تقسیم برای چند جمله ایها روی یک هیات، حوزه های تجزیه یکتا، حوزه ایدهآل اصلی و حوزه اقلیدسی.
        }}
        },{{
            تحقیق در عملیات 1
        }{
            22882
        }{
            4
        }{
            جبر خطی 1
        }{}{{
            آشنایی با زمینه های تحقیق در عملیات، انواع مدلهای ریاضی، برنامه ریزی خطی (مدل بندی، روشهای ترسیمی، سیمپلکس اولیه و دوگان، دو فازی M بزرگ، دوگانی و نتایج آن، آنالیز حساسیت) شبکه ها و مدل حمل و نقل و تخصیص، سایر مدلهای مشابه، آشنایی با برنامه ریزی متغیرهای صحیح، آشنایی با برنامه ریزی پویا، آشنایی با برنامه ریزی غیر خطی، آشنایی با مدلهای احتمالی.
        }}
        },{{
            فرایندهای تصادفی
        }{
            22635
        }{
            4
        }{
            احتمال و کاربرد آن
        }{{
            1- آشنایی با طیف وسیعی از فرایندهای تصادفی و ایجاد توانایی در دانشجو جهت ساختن مدلهای تصادفی، یادگیری مفاهیم نظری و کاربردی تاکید بر کاربردهای فرایندهای تصادفی.
        }}{{
            تعاریف و مفاهیم پایهای در مورد فرایند تصادفی، قدم زدن تصادفی، تعاریف و مفاهیم پایهای در مورد مارتینگل و زیر مارتینگل، فرایند مارکف، فرایندهای گاوسی، آشنایی با حرکت براونی و کاربردهای آن، فرایند پواسن، زمانهای رسیدن رویدادها، زمانهای بین رویدادها، تعاریف. و مفاهیم پایهای در مورد زنجیره ای مارکف، ماتریس انتقال حالت، معادلات چپمن –کلموگرف، انواع حالات، رفتار مجانبی زنجیر مارکف، احتمالهای حدی و ایستا، زنجیر مارکف بازگشت پذیر در زمان، زنجیرهای مارکف پیوسته در زمان، معادلات کلموگرف رو به عقب در زمان و رو به جلو در زمان، کاربردهای فرایندهای تصادفی مانند نظریه صف.
        }}
        },{{
            تحلیل رگرسیون
        }{
            22614
        }{
            4
        }{
            احتمال و کاربرد آن و جبر خطی 1
        }{{
            1- ایجاد توانایی در دانشجو جهت آزمونهای پیشرفته فرض آماری برای برآورد پارامترهای مدلهای آماری گسسته و پیوسته با تاکید بر مدلهای خطی، تعیین میزان کیفیت این پارامترها و آشنایی با نرم افزارهای جدید در این مورد و استفاده از آن.
        }}{{
            آشنایی با آماره ها، فرمهای درجه دوم از متغیرهای تصادفی و توزیعهای آنها، ماتریسهای واریان و کوواریانس، رگرسیون خطی یک متغیره و چند متغیره، برآورد پارامترها و آزمونهای فرض برای مدل با رتبه کامل، روشهای کمترین مربعات و حداکثر احتمال، براورد پارامترها و آزمونهای فرض برای مدل با رتبهٔ ناکامل، سنجش کیفیت رگرسیون، مدلهای قطعی و مقایسهٔ آنها با مدلهای تصادفی، مدلهای رگرسیون با متغیرهای مجازی، اندرکنش در رگرسیون، تحلیلهای واریانس یک طرفه و دو طرفه و کوواریانس، پیش بینی بر اساس رگرسیون خطی، آشنایی با روشهای رگرسیون غیر خطی مانند رگرسیون لجیستیکی و رگرسیون پواسن آشنایی با مدلهای خطی تعمیم یافته.
        }}
        },{{
            آنالیز ریاضی 2
        }{
            22326
        }{
            4
        }{
            انالیز ریاضی 1
        }{{
            1- تدریس آنالیز ریاضی توابع چند متغیره حقیقی و انتگرال لبگ به گونه ای که در این درس مطالب مورد نیاز به عنوان پیشنیاز دروس مختلف ریاضی پوشانده شده باشد.
        }}{{
            تبدیل خطی و خواص آنالیزی آن، مشتق تابع چند متغیره، قاعده زنجیری مشتقات پارهای، قضیه نگاشت معکوس، قضیه تابع ضمنی، قضیه رتبه، قضیه های ماکزیمم و مینیمم، قضیه لاگرانژ. اندازه و انتگرال لبگ روی R و Rn، قضیه های همگرایی معروف، مقایسه انتگرال لبگ و ریمان و قضیه ریمان–لبگ، قضیه فوبینی، تعویض متغیر در انتگرال چندگانه.
        }}
        },{{
            توپولوژی 1
        }{
            22556
        }{
            4
        }{
            آنالیز ریاضی 1
        }{{
            1- تدریس مفاهیم پایهای در فضاهای توپولوژیک به گونه ای که در این درس مطالب مورد نیاز به عنوان پیشنیاز دروس مختلف ریاضی پوشانده شده باشد
        }}{{
            مقدمات نظریه مجموعه ها، فضای توپولوژیک و فضای متریک، پایه و زیر پایه، پیوستگی، توپولوژی حاصلضرب، زیر فضای توپولوژیک، توپولوژی خارج قسمت، همگرایی به روش تور یا فیلتر، انواع همبندی و قضایای مربوطه، انواع فشردگی و قضایای مربوطه، اصول شمارش پذیری و جدا سازی، قضیه تیخونف، لم اوریسن، قضیه توسیع تیتسه، قضیه متری سازی اورپسن، مفهوم فشرده سازی، فضای متریک کامل، فضای متریک تابعی، همپیوستگی و قضایای آرزلا – آسکولی. لازم به توضیح است که در کمیته برنامه ریزی پیشنهاد شد که مفهوم همگرایی نیز معرفی گردد.
        }}
        },{{
            سریهای زمانی
        }{
            22628
        }{
            4
        }{
            آمار و کاربرد آن
        }{{
            ایجاد توانایی در دانشجو جهت پیش بینی آینده بر اساس اطلاعات گردآوری شده از گذشته تا زمان حال آشنایی با مدلهای گوناگون متداول برای این پیش بینی و آشنایی با نرم افزارهای جدید در این مورد و استفاده از آن.
        }}{{
            مفاهیم مقدماتی و پایهای در ارتباط با سریهای زمانی گسسته و پیوسته، فرایندهای ایستا و غیر ایستا، تابع خود هم بستگی، تابع خود همبستگی جزئی، تابع خود همبستگی وارون، فرایند اتورگرسیو و بررسی شرایط ایستایی آن، فرایند میانگین متحرک MA و بررسی شرایط وارون پذیری آن، مدل سازی و پیش بینی با استفاده از فرایندهای ARMA، ARIMA، SARIMA روش باکس – جنکینز، مدلهای تابع تبدیل، تحلیل دخالت، تحلیل طیفی سریهای زمانی ریال قضیه تجزیه والد، آشنایی با مدلهای فضای حالت، آشنایی با سریهای زمانی چند متغیره.
        }}
        },{{
            نظریه گراف و کاربرد آن
        }{
            22162
        }{
            4
        }{
            ریاضیات گسسته
        }{}{{
            آشنایی با مفاهیم مربوط به گراف از قبیل درجه راس، یکریختی گرافها، زیر گرافها، دنباله درجه ها، گرافهای همبند، راسها و یالهای برشی، گرافهای خالص، گرافهای جهت دار و کاربرد. آشنایی با الگوریتمها: پیچیدگی الگوریتمی، الگوریتم جستجو، الگوریتم مرتب کردن، مقدمهای بر NP تمامیت، الگوریتم آزمند، چگونگی معرفی یک گراف به کامپیوتر و. درختها و الگوریتمهای مربوط به آنها از قبیل DFS. BFS مینیمم درخت فراگیر و کاربردهای هر کدام از آنها. مسیرها و فاصله ها در گراف، گراف جهتدار فعالیت و مسیرهای بحرانی، کدهای تصحیح کننده خطا به عنوان یک کاربرد. شبکه ها و قضیه شار ماکزیمم و برش مینیمم، پیچیدگی الگوریتم شار ماکزیمم و برش مینیمم، همبندی و همبندی یالی، قضیه منگر و کاربردهای آن. مقدمهای بر تطابق در گرافها، تطابق ماکزیمم در گرافهای دو بخشی و کلاً در گرافها، تجزیه به تطابقهای کامل، کاربردها مثلاً در طرحهای بلوکی. گرافهای اویلری و مساله پستچی چینی، گرافهای اویلری جهت دار. آشنایی با گرافهای هامیلتونی و کاربرد آن در مساله فروشنده دوره گرد. گرافهای مسطح و الگوریتمی برای آزمون مسطح بودن، اعداد متقاطع، ضخامت و گونا در گرافها، ماینورها، رنگ آمیزیهای مختلف در گرافها، چند جمله ای رنگی، مساله 4- رنگ رنگ آمیزی یالی و کاربردها. گرافهای جهت دار، مسائل و کاربردهای آنها.
        }}
        },{{
            نظریه مقدماتی احتمال
        }{
            22338
        }{
            4
        }{
            احتمال و کاربرد آن، آنالیز ریاضی 2
        }{}{{
            انگیزه های نظریه احتمال و یادآوری مفاهیم مقدماتی، آزمایش پرتاب سکه، توابع رادماخر، قانون اعداد بزرگ برای دنباله های برنولی، نظریه مقدماتی اندازه، میدان سیگمایی، قضیه توسیع کاراتئودوری، استقلال، قضایای برل – کانتلی، امید ریاضی برای متغیرهای تصادفی ساده، تقریب متغیرهای تصادفی با متغیرهای تصادفی ساده، قوانین ضعیف و قوی اعداد بزرگ، همگرایی در توزیع، تابع مشخصه، قضیه حد مرکزی.
        }}
        },{{
            توابع مختلط 1
        }{
            22335
        }{
            4
        }{
            آنالیز ریاضی 1
        }{}{{
            دستگاه اعداد مختلط و کره ریمان، تبدیلات موبیوس، توابع تحلیلی، معادلات کوشی – ریمان، انتگرال گیری، قضیه کوشی، فرمول انتگرال کوشی و نتایج آن، اصل ماکسیمم، سریهای توانی، سری تیلور و لوران، تکینه ها، حساب مانده ها و کاربرد آن، نظریه نگاشتهای همدیس، خانواده نرمال، قضیه نگاشت ریمان، فرمول شوارتس – کریستوفل، توابع هارمونیک، مساله دیریکله، فرمول انتگرال پواسون.
        }}
        },{{
            آنالیز فوریه و کاربرد آن
        }{
            آنالیز فوریه و کاربرد آن
        }{
            4
        }{
            آنالیز ریاضی 2
        }{}{{
            یادآوری نظریه مقدماتی اندازه و انتگرال شامل قضایای همگرای L و L، هسته دیریکله و فیر، قضیه پارسوال، سری فوریه در ابعاد بالا، انتگرال فوریه، قضیه پلانشرل و کاربردهای آن، انتگرال فوریه در ابعاد بالا، کاربرد آنالیز فوریه در احتمال و معادلات دیفرانسیل با مشتقات جزئی.
        }}
        },{{
            آنالیز تابعی مقدماتی
        }{
            22475
        }{
            4
        }{
            آنالیز ریاضی 2
        }{}{{
            فضاهای باناخ و هیلبرت، عملگرهای خطی کراندار و بیکران، عملگرهای خود الحاق عملگرهای فشرده و خواص ابتدایی آن، نظریه طیفی عملگرها، کاربردها عملگرها در معادلات دیفرانسیل و انتگرال.
        }}
        },{{
            ترکیبیات و کاربردهای آن
        }{
            22118
        }{
            4
        }{
            ریاضیات گسسته
        }{}{{
            ترکیبات چیست؟ مثالهایی از قبیل پوشش کامل صفحه شطرنج، برش مکعب، مربعهای جادوئی، مساله 4- رنگ، مساله 36 افسر اویلر، مساله کوتاهترین مسیر، بازی نیم و غیره. اصل لانه کبوتری با صورت ساده و با صورت قوی، یک قضیه رمزی (Ramsey) به عنوان کاربرد. جایگشتها و ترکیبها روی مجموعه ها و چند – مجموعه ها با کاربردهای آنها. الگوریتمهای تولید جایگشتها و ترکیبها، ترتیبهای جزئی، رابطه های هم ارزی و کاربرد آنها. قضیه های دو جمله ای و چند جمله ای، قضیه دو جمله ای نیوتون، بررسی بیشتر از مجموعه های مرتب جزئی و کاربردهایشان. رابطه های بازگشتی و توابع مولد با کاربرد. دنباله های شمارشی خاص، اعداد کاتالان، دنباله های تفاضلی و اعداد استرلینگ، افراز اعداد. کاربردها. سیستم نمایندگی متمایز و مساله ازدواج پابرجاه، کاربردهای مختلف در انتخاب شغل، پذیرش دانشگاهی و غیره. اشنائی مختصر با طرحهای ترکیبیاتی از قبیل طرح بلوکی، سیستم سه گانه اشتاینر، مربعهای لاتین و کاربرد آنها. جایگشتها و گروه ها تقارن، قضیه برنساید و فرمول شمارش پولیا و کاربردهای ترکیبیاتی آن.
        }}
        },{{
            جبر 2
        }{
            22218
        }{
            4
        }{
            جبر 1
        }{}{{
            1ـ گروه ها، عمل گروه بر یک مجموعه و قضایای مربوط، قضیه شمارشی برنساید، قضایای سیلو، P – گروه های، گروه های ساده، مثالهای متنوع از گروه ها مانند گروه های دو وجهی، گروهای ماتریسی و گروه های تقارن، رشته گروه ها و قضیه ژوردان- هولدر، گروه های حل پذیر و پوچتوان، ساختار گروه های از مرتبه حاصلضرب دو عدد اول. 2- حلقه ها و میدانها، جمع و ضرب ایده آلها، رادیکال یک ایده آل، رادیکال پوچ و رادیکال جیکوبسن در حلقه های جابه جایی، حلقه خارج قسمت نسبت به یک مجموعه ضربی و موضعی سازی، حلقه های موضعی، حلقه هیا نوتری و آرتینی، اعداد صحیح جبری، حلقه های تقسیم، مثالهای مهم و روشهای مختلف برای ساختن میدانها، توسیعهای میدانی، فرمول برج، چند جمله ایهای تحویل ناپذیر روی میدانها، آزمون آیزنشتاین، میدانهای بسته جبری.
        }}
        },{{
            جبر 3
        }{
            22209
        }{
            4
        }{
            جبر 1
        }{}{{
            نظریه گالوا شامل توسیع میدانها، توسیعهای جبری، نرمال، جدایی پذیر، گالوا، قضیه اساسی نظریه گالوا، میدانهای متناهی، گسترشهای دابره بر، دوری، کومر، کاربردهایی مانند محاسبه گروه گالوای چند جمله های درجه سوم و چهارم، ساختهای خط کش و پرگاری، حل پذیری با رادیکالها.
        }}
        },{{
            نظریه اعداد
        }{
            22215
        }{
            4
        }{
            جبر 1
        }{}{{
            مقدمات جبری (تجزیه یکتا در Z در [k[x و بطور کلی در PIDها)، نتایج یکتایی تجزیه (شامل مطالعه مقدماتی توزیع اعداد اول)، آشنایی با توابع حسابی (حاصلضرب دیریکله، قضیه وارونسازی موبیوس، توابع حسابی خاص)، هم نهشتیها در Z (آشنایی با معادلات دیوفانتوسی، معادلات هم نهشتی خطی، قضیه باقیمانده چینی)، ساختار گروه یکالهای حلقه Z/nZ، تقابل مربعی (صورتهای گوناگون قانون تقابل مربعی، ارائه چند اثبات متفاوت)، مجموعه های گاوسی مربعی (همراه کاربردهایی مانند اثبات مجدد تقابل مربعی، آشنایی با اعداد جبری و اعداد صحیح جبری)، اشنایی با میدانهای متناهی (همراه کاربردهایی در نظریه اعداد)، مجموعه های گاوسی و مجموعه های ژاکوبی (همراه کاربردهایی مانند محاسبه تعداد جوابهای برخی معادلات در Fp)، تجزیه اولهای گویا در [Z[w]، Z[i (همراه کاربردهایی مانند قضیه دو مربع و مشابه آن).
        }}
        },{{
            هندسه جبری مقدماتی
        }{
            22532
        }{
            4
        }{
            جبر 1
        }{}{{
            صفحه مستوی و تصویری روی یک میدان، خمهای جبری، قضیه بزو، نقاط ساده و تکین، خمهای درجه 3، واریته های آبلی و گروهی، سری توانی صوری، بسط در همسایگی نقاط ساده، شاخه، نقاط نوعی، صفر و قطب، دیفرانسیل، گونه، قضیه ریمان – رخ، روشهای نوین در هندسه جبری مانند شما (Scheme) و طیف یک حلقه با مقدمات جبری لازم.
        }}
        },{{
            نظریه مقدماتی معادلات دیفرانسیل عادی
        }{
            22384
        }{
            4
        }{
            آنالیز ریاضی 1، جبر خطی 1
        }{}{{
            قضایای وجود یکتایی دستگاه ها، وابستگی به شرایط اولیه و پارامتر، شاره و فضای فاز، ارتباط با مکانیک نیوتنی، دستگاه های خطی و صورتهای متعارف، پایداری در دستگاه های خطی، دستگاه های غیر خطی، خطی سازی، بررسی نقاط تکین و جوابهای تناوبی، دستگاه های تناوبی، دستگاه های تناوبی و نظریه فلوکه، تابع لیاپونف، خمینه های پایدار و ناپایدار، قضیه پوانکاره – بندیکسون، دستگاه های لینارد و معادله ون در پل، نظریه مقدماتی انشعاب و خمینه مرکزی.
        }}
        },{{
            معادلات دیفرانسیل با مشتقات جزئی
        }{
            22395
        }{
            4
        }{
            آنالیز ریاضی2
        }{}{{
            1- معادلات خطی مرتبه دوم و روش منحنی مشخصه، مفاهیم و تعاریف مقدماتی، دسته بندی معادلات خطی مرتبه دوم، روش دالامبر برای حل معادلات موج همگن و غیر همگن در بازه های نیمه نامتناهی و نامتناهی. 2- سری فوریه، سری فوریه کسینوسی و سینوسی، نامساوی بسل، اتحاد پارسوال، فرم مختلط سری فوریه، همگرایی نقطهای، یکنواخت و در میانگین. مشتق و انتگرال سری فوریه. 3- روش جدا سازی متغیرها، وجود و یکتایی معادلات فنر مرتعش و حرارت روی یک بازه متناهی، معادلات حرارت و فنر مرتعش غیر همگن با شرایط مرزی و اولیه غیر همگن. 4- مساله استورم – لیوویل، مقادیر و توابع ویژه، بسط توابع ویژه، همگرایی در میانگین، نامساوی بسل، اتحاد پارسوال، اتحاد لاگرانژ، تعریف و نحوه ساختن تابع گرین، مسایل با شرط مرزی غیر همگن، مسایل مقدار ویژه و تابع گرین. 5- مسایل با شرط مرزی، مساله دیریکله و نویمن، اصل ماکسیمم و مینیمم، یکتایی جواب مساله دیریکله و پیوستگی جوابها نسبت به شرط اولیه، مساله دیریکله و و نویمن روی دایره، مساله دیریکله روی حلقه های دوار، مساله دیریکله و نویمن روی مستطیل. 6- مسایل در ابعاد بالاتر، مساله دیریکله در مکعب، استوانه و کره، معادلات حرارت و موج، غشای مرتعش، جریان حرارت یک صفحه مستطیلی و یک مکعب مستطیل، معادلات موج در بعد 3، روش توابع ویژه 7- تابع گرین، تابع دلتا، تابع گرین، روش تابع گرین مساله دیریکله برای عملگر لاپلاس و هلم هولتز، روش توابع ویژه، مساله با ابعاد بالاتر، مساله نویمن. 8- تبدیلهای انتگرالی، تبدیلهای فوریه، لاپلاس، هنکل و ملین، خواص و کاربرد آنها در حل معادلات حرارت موج و لاپلاس در نواحی نیمه نامتناهی و نامتناهی.
        }}
        },{{
            آشنایی با سیستمهای دینامیکی
        }{
            22375
        }{
            4
        }{
            آنالیز ریاضی 1، جبر خطی 1
        }{}{{
            مباحثی از دینامیک توابع از بازه به بازه و دایره به دایره، عدد چرخشی، قضیه دانژوا، خانواده توابع درجه دوم از بازه به بازه، دینامیک نمایدن، آشوب، پایداری ساختاری، قضیه شارکوفسکی، نظریه انشعاب، دینامیک توابع مختلط، خودریختیهای چنبرهای و نعل اسب، آنتروپی.
        }}
        },{{
            مبانی ریاضیات
        }{
            22142
        }{
            4
        }{
            ندارد
        }{}{{
            مفاهیم ابتدایی نظریه مجموعه ها مانند اجتماع، اشتراک، مجموعه توان و…، بیان اصول نظریه مجموعه ها، ساختن اعداد طبیعی، صحیح، گویا و حقیقی، معرفی برشهای ددکیند و دنباله های کوشی، اصل انتخاب و بعضی معادلهای مهم آن مانند لم زورن و کاربرد آن در اثبات قضایای اساسی ریاضیات، اعداد اصلی و ترتیبی. توضیح: دانشجو نمی تواند در هر دو درس مبانی ریاضیات و نظریه مجموعه ها واحد درسی کسب کند.
        }}
        },{{
            نظریه مقدماتی مجموعه ها
        }{
            22133
        }{
            4
        }{
            ریاضی عمومی 1
        }{}{{
            مفاهیم ابتدایی نظریه مجموعه ها، بیان اصول نظریه مجموعه ها، ساختن اعداد طبیعی، صحیح، گویا و حقیقی، اصل انتخاب و بعضی معادلهای مهم آن مانند لم زورن و کاربرد آن در اثبات قضایای اساسی ریاضیات، حساب اعداد اصلی و ترتیبی، استقرار فرانهایی، ساختار اعداد حقیقی در رابطه با فرضیه پیوستار، معرفی جهان گودلی، اصل V=L، اثبات سازگاری اصل انتخاب و فرضیه پیوستار با اصول نظریه مجموعه ها. توضیح: دانشجو نمی تواند در هر دو درس مبانی ریاضیات و نظریه مجموعه ها واحد درسی کسب کند.
        }}
        },{{
            منطق ریاضی
        }{
            22144
        }{
            3
        }{
            مبانی ریاضیات یا نظریه مقدماتی مجموعه ها
        }{}{{
            زبان منطق گزاره ها، نحو و معناشناسی منطق گزاره ها، استنتاج طبیعی، قضایای صحت و تمامیت در منطق گزاره ها، تصمیم پذیری منطق گزاره ها، زبان منطق مرتبه اول، نحو و معناشناسی منطق مرتبه اول، استنتاج طبیعی، قضایای صحت و تمامیت در منطق مرتبه اول، قضیه فشردگی، قضایای افزایشی و کاهشی لونهایم – اسکولم و کاربردهای مختلف آن، حساب و آنالیز غیر استاندارد، مفاهیم قضایای ابتدایی نظریه مدلها مانند مفاهیم زیر مدل، زیر مدل مقدماتی، همریختی و یکریختی بین مدلها و … معرفی کلی زبان و منطق مرتبه دوم.
        }}
        },{{
            هندسه دیفرانسیل مقدماتی
        }{
            22542
        }{
            4
        }{
            آنالیز ریاضی 2
        }{}{{
            نظریه خمها در Rn، کنج فرنه، نمایش موضعی خم در R3 در همسایگی یک نقطه، قضیه بنیادی خمها، برخی قضایای سرتاسری در R2 مانند قضیه مماس گردان. نظریه موضعی رویه ها در R3، نگاشت گاوس، فرمهای بنیادی اول و دوم، انحناهای اصلی، انحنای گاوسی و میانگین، رویه های خط کشی شده، رویه های مینی مال، معادلات گاوس – کداتسی – مایناردی، قضیه گاوس، هندسه ذاتی رویه ها و هندسه ریمانی دوبعدی، مشتقگیری همورد، زئودزیکی، قضیه گاوس – بونه.
        }}
        },{{
            هندسه هذلولی
        }{
            22584
        }{
            4
        }{
            توابع مختلط 1
        }{}{{
            مقدمه تاریخی در مورد پیدایش هندسه های غیر اقلیدسی، مدلهای هندسه هذلولی، هندسه هذلولی تحلیلی بر اساس یکی از مدلها مانند مدل پوانکاره، نگاشتهای موبیوس، مفاهیم طول، زاویا و مساحت در هندسه هذلولی، گروه ایزومتری، مثلثات هذلولی، تبدیلات بیضوی، هذلولی و سهموی، ناحیه های بنیادی، مجموعه های حدی گروه های ایزومتری، هندسه هذلولی در ابعاد 3 به بالا و در صورت امکان مباحث پیشرفتهتر.
        }}
        },{{
            آشنایی با توپولوژی جبری
        }{
            22565
        }{
            4
        }{
            توپولوژی 1، جبر 1
        }{}{{
            آشنایی با مباحثی از توپولوژی جبری مانند گروه بنیادی، فضای پوششی و نظریه همولوژی سادکی (simplicial) با تاکید بر کاربردهای ملموس چون قضیه ژردان، قضیه نقطه ثابت براوئر، شاخص اویلر، قضیه برسوک – اولام، درجه، قضیه لفشتز و نظریه مقدماتی گره ها.
        }}
        },{{
            توپولوژی دیفرانسیل مقدماتی
        }{
            22564
        }{
            4
        }{
            توپولوژی 1، آنالیز ریاضی 2
        }{}{{
            اشنایی در سطح مقدماتی با منتخبی از مفاهیم توپولوژی دیفرانسیل مانند تراگذری (transversality) نظریۀ مرس، عدد تقاطع، عدد اویلر، عدد لفشتز، جراحی و کاربردهایی چون قضیه ژردان، قضیه های براوئر، قضیه برسوک – اولام، قضیه لفشتز، قضیه پوانکاره – هوپف، قضیه درجه هوپف.
        }}
        },{{
            برنامه نویسی پیشرفته
        }{
            22815
        }{
            4
        }{
            برنامهسازی کامپیوتر و ریاضیات گسسته
        }{}{{
            روشهای حل مساله از قبیل ذهنی و موازی، انواع برنامه سازی (عملیاتی و موضعی و یا رویهای شی گرا) مفهوم داده مجرد، انواع داده ها شامل رکورد و نشانه، STACK انواع صف، درختها و درخت دودوئی، درخت دودوئی، درخت دودوئی جستجو، کاربرد درخت در برخی مسائل نمونه. اثبات صحت الگوریتمها، اثبات توقف و عدم توقف، پیچیدگی عملیات حافظه، مفاهیم اساسی چرخه عمر تولید نرم افزار، یک زبان برنامه نویسی، تهیه و اجرا پروژه های عملی در این زبان در خصوص مطالب درس.
        }}
        },{{
            ساختمان داده ها
        }{
            code
        }{
            4
        }{
            برنامهنویسی پیشرفته و ریاضیات گسسته
        }{}{{
            مفاهیم کلی، رابطه بین ساختمان داده ها و الگوریتم، ساختمانهای ایستا، مروری بر آرایه ها، ماتریسها ‚ ماتریسهای خلوت ‚ نمایش آرایه ها، ساختمانهای نیمه ایستا، مروری بر انباره ها و صفها، کاربرد آنها (محاسبه عبارت جبری)، ساختمانهای پویا، لیستها پیوندی، خطی، حلقوی، با پیوند مضاعف، چند پیوندی، روش نمایش و کاربرد لیستهای پیوندی، الگوریتمهای بازگشتی، درختها و پیمایش آنها، مروری بر درخت دودوئی و نمایش آن، تبدیل درخته به درخت دودوئی، پیمایش پیش ترتیب و میان ترتیب و پس ترتیب، کاربرد درختها، انواع درختها (درخت تصمیم گیری، درخت جستجو، درخت بازی و غیره). توازن درختها، روشهای نمایش، گرافها و نمایش آنها، گراف جهت دار، گراف، روشهای پیمایش (جستجوی ژرفائی، روشهای حل مسئله شامل تقسیم و تسخیر، الگوریت حریص دایسترا، الگوریتمهای احتمالی، مسئله کوله پشتی و برنامه ریزی پویا، مثالهای متنوع شامل مرتب کردن و جستجو (جستجوی پراکنده، توابع درهم سازی، مرتب کردن سریع، ادغامی، هرمی، مرتب کردن خارجی) و مقایسه پیچیدیگی آنها، پردازش لیستها و رشته ها.
        }}
        },{{
            تحقیق در عملیات 2
        }{
            22901
        }{
            4
        }{
            تحقیق در عملیات
        }{}{{
            مروری بر برنامهریزی خطی به روش برداری و دوگانی.
            ,
            برنامه ریزی متغیرهای صحیح: مدل بندی مسائل یک – صفر، حل مسائل یک – صفر به روش شمارش صریح و ضمنی، مدل بندی مسائل متغیرهای صحیح، حل مدلهای متغیر صحیح به روشهای شاخه و کران و صفحه برشی.
            ,
            برنامه ریزی پویا: اصول و تعاریف، مدل بندی مسائل غیر احتمالی، معادلات بازگشتی، روشهای حل مدلهای با متغیر وضعیت ناپیوسته، روش حل مدلها با متغیر وضعیت پیوسته، موارد کاربردی.
            ,
            برنامه ریزی غیر خطی: اصول کلاسیک بهینه سازی، مسائل بدون قید، مسائل قید دار لاگرانژ
            ,
            برنامه ریز درجه دوم، برنامه ریزی مسائل جداپذیر، روشهای جستجو.
            ,
        }}
        }
    }
\end{document}